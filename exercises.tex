% Notes and exercises on Category Theory
\documentclass[letterpaper,12pt]{article}
\usepackage{amsmath,amssymb,amsthm,enumitem,fourier,diagrams,tikz-cd}
\usepackage[hidelinks]{hyperref}

\newcommand{\N}{\mathbb{N}}

\newcommand{\iso}{\cong}
\newcommand{\eq}{\sim}
\newcommand{\eqv}{\simeq}
\newcommand{\mono}{\rightarrowtail}
\newcommand{\epi}{\twoheadrightarrow}

\newcommand{\meet}{\wedge}
\newcommand{\bigmeet}{\bigwedge}
\newcommand{\join}{\vee}
\newcommand{\bigjoin}{\bigvee}
\newcommand{\compl}{\lnot}
\newcommand{\obj}{{\cdot}}
\newcommand{\after}{\circ}
\newcommand{\limit}{\varprojlim}
\newcommand{\colimit}{\varinjlim}
\newcommand{\bigunion}{\bigcup}
\newcommand{\eval}{\epsilon}
\newcommand{\term}{{!}}

\DeclareMathOperator{\dom}{dom}
\DeclareMathOperator{\cod}{cod}
\DeclareMathOperator{\fdom}{\mathbf{dom}}
\DeclareMathOperator{\fst}{fst}
\DeclareMathOperator{\snd}{snd}
\DeclareMathOperator{\Hom}{Hom}
\DeclareMathOperator{\Sub}{Sub}
\DeclareMathOperator{\Diagrams}{\mathbf{Diagrams}}
\DeclareMathOperator{\pow}{\mathcal{P}}
\DeclareMathOperator{\ult}{Ult}
\DeclareMathOperator{\up}{\uparrow}

\newcommand{\pair}[2]{\langle{#1},{#2}\rangle}
\newcommand{\copair}[2]{[{#1},{#2}]}
\newcommand{\comp}[1]{\overline{#1}}
\newcommand{\inv}[1]{#1^{-1}}
\renewcommand{\star}[1]{#1^{*}}
\newcommand{\cat}[1]{\mathbf{#1}}
\newcommand{\dual}[1]{#1^{\mathrm{op}}}
\newcommand{\arr}[1]{#1^{\rightarrow}}
\newcommand{\under}[1]{|{#1}|}
\newcommand{\gen}[1]{\langle{#1}\rangle}
\newcommand{\curry}[1]{\lambda{#1}}
\newcommand{\uncurry}[1]{\overline{#1}}
\newcommand{\pull}[1]{#1^{*}}
\newcommand{\fin}[1]{#1_{\mathrm{fin}}}

\newcommand{\A}{\cat{A}}
\newcommand{\B}{\cat{B}}
\newcommand{\C}{\cat{C}}
\newcommand{\Cop}{\dual{\C}}
\newcommand{\D}{\cat{D}}
\newcommand{\E}{\cat{E}}
\newcommand{\J}{\cat{J}}
\newcommand{\Rel}{\cat{Rel}}
\newcommand{\Relop}{\dual{\Rel}}
\newcommand{\Sets}{\cat{Sets}}
\newcommand{\Setsop}{\dual{\Sets}}
\newcommand{\Setsp}{\Sets_*}
\newcommand{\Setsf}{\fin{\Sets}}
\newcommand{\Par}{\cat{Par}}
\newcommand{\Mon}{\cat{Mon}}
\newcommand{\Grp}{\cat{Groups}}
\newcommand{\Pre}{\cat{Pre}}
\newcommand{\Pos}{\cat{Pos}}
\newcommand{\Types}{\C(\lambda)}
\newcommand{\Cat}{\cat{Cat}}
\newcommand{\BA}{\cat{BA}}
\newcommand{\Ord}{\cat{Ord}}
\newcommand{\Ordf}{\fin{\Ord}}

\newcommand{\powBA}{\pow^{\BA}}

% Arrows
\newarrow{Mono} >--->
\newarrow{Epi} ----{>>}

% Theorems
\theoremstyle{definition}
\newtheorem*{exer}{Exercise}

\theoremstyle{remark}
\newtheorem*{rmk}{Remark}

\newtheoremstyle{direction}{0.5em}{0.5em}{}{}{}{}{0.5em}{}
\theoremstyle{direction}
\newtheorem*{fwd}{\(\implies\)}
\newtheorem*{bwd}{\(\impliedby\)}

% Meta
\title{\textit{Category Theory}\\Notes and Exercises}
\author{John Peloquin}
\date{}

\begin{document}
\maketitle

\section*{Introduction}
This document contains notes and exercises from~\cite{awodey10}.

\section*{Chapter 1}
\begin{exer}[1]
In~\(\Rel\), let the objects be sets and the arrows be relations between sets,\footnote{An arrow \(A\to B\) between sets \(A\)~and~\(B\) is understood as a triple \((R,A,B)\) with \(R\subseteq A\times B\).} with identities and composites defined as usual for relations.
\begin{enumerate}[itemsep=0pt]
\item[(a)] \(\Rel\)~is a category.
\item[(b)] Let \(G:\Sets\to\Rel\) map sets to themselves and functions to their graphs, so \(G(A)=A\) and
\[G(f:A\to B)=\{\,\pair{x}{f(x)}\mid x\in A\,\}\subseteq A\times B\]
Then \(G\)~is a functor.
\item[(c)] Let \(C:\Relop\to\Rel\) map sets to themselves and relations to their inverses, so \(C(A)=A\) and
\[C(R\subseteq A\times B)=\inv{R}=\{\,\pair{y}{x}\mid\pair{x}{y}\in R\,\}\subseteq B\times A\]
Then \(C\)~is a functor.
\end{enumerate}
\begin{proof}\
\begin{enumerate}[itemsep=0pt]
\item[(a)] We must verify that composition of relations is associative and unital. Suppose \(R\subseteq A\times B\), \(S\subseteq B\times C\), and \(T\subseteq C\times D\). For \(\pair{w}{z}\in A\times D\), by the definition of composition we have
\begin{align*}
\pair{w}{z}\in(T\after S)\after R&\iff\exists x\in B[\pair{w}{x}\in R\land\pair{x}{z}\in T\after S]\\
								&\iff\exists x\in B,y\in C[\pair{w}{x}\in R\land\pair{x}{y}\in S\land\pair{y}{z}\in T]\\
								&\iff\exists y\in C[\pair{w}{y}\in S\after R\land\pair{y}{z}\in T]\\
								&\iff\pair{w}{z}\in T\after(S\after R)
\end{align*}
So \((T\after S)\after R=T\after(S\after R)\). It is immediate that \(R\after 1_A=R=1_B\after R\), where \(1_X\)~denotes the identity relation on~\(X\).
\item[(b)] By construction, \(G\)~maps objects to objects and arrows to arrows, and \(G(f:A\to B)\)~is an arrow from \(G(A)=A\) to \(G(B)=B\). Clearly \(G(1_A)=1_A=1_{G(A)}\). If \(f:A\to B\) and \(g:B\to C\), then \((g\after f)(x)=z\) if and only if \(f(x)=y\) and \(g(y)=z\), so \(G(g\after f)=G(g)\after G(f)\).
\item[(c)] Recall for a relation \(R\subseteq A\times B\), \(R\)~is represented as an arrow in~\(\Rel\) by the triple~\((R,A,B)\), and in~\(\Relop\) by the triple~\((R,B,A)\), where it is denoted by~\(\star{R}\).\footnote{Importantly, \(R\)~is \emph{not} represented in~\(\Relop\) by~\((\inv{R},B,A)\). The arrow is reversed by swapping the domain and codomain, but the underlying relation (set of ordered pairs) is unchanged.} So in~\(\Rel\), \(\dom R=A\) and \(\cod R=B\), whereas in~\(\Relop\), \(\dom\star{R}=B=\star{B}\) and \(\cod\star{R}=A=\star{A}\), where \(R\)~and~\(\star{R}\) are here treated as arrows.

By construction, \(C\)~maps objects to objects and arrows to arrows. Now \(C((R,B,A))=(\inv{R},B,A)\), so \(C\)~preserves domains and codomains. Also
\[C(1_{\star{A}})=C(\star{1_A})=\inv{1_A}=1_A=1_{C(\star{A})}\]
For \(S\subseteq B\times C\),
\[C(\star{R}\after\star{S})=C(\star{(S\after R)})=\inv{(S\after R)}=\inv{R}\after\inv{S}=C(\star{R})\after C(\star{S})\qedhere\]
\end{enumerate}
\end{proof}
\end{exer}

\begin{exer}[2]\
\begin{enumerate}[itemsep=0pt]
\item[(a)] \(\Rel\iso\Relop\)
\item[(c)] For any set~\(X\) with powerset~\(P(X)\), \(P(X)\iso\dual{P(X)}\) as poset categories.
\end{enumerate}
\begin{proof}\
\begin{enumerate}[itemsep=0pt]
\item[(a)] The functor in Exercise~1(c) is its own inverse, hence is an isomorphism.
\item[(c)] Recall in~\(P(X)\) there exists a unique arrow \(A\to B\) if and only if \(A\subseteq B\), hence in~\(\dual{P(X)}\) there exists a unique arrow \(A\to B\) if and only if \(A\supseteq B\).

For \(A\subseteq X\), write \(\comp{A}=X-A=\{\,x\in X\mid x\not\in A\,\}\). Define \(C:\dual{P(X)}\to P\) by \(C(A)=\comp{A}\) and
\[C(A\to B)=\comp{A}\to\comp{B}=C(A)\to C(B)\]
which is well defined since \(A\supseteq B\) if and only if \(\comp{A}\subseteq\comp{B}\). Clearly \(C\)~maps objects to objects and arrows to arrows, and also preserves domains and codomains. Substituting \(A\)~for~\(B\) above shows that \(C\)~preserves identities. For \(X\supseteq A\supseteq B\supseteq D\),
\[C(A\to B\to D)=\comp{A}\to\comp{B}\to\comp{D}=C(A)\to C(B)\to C(D)\]
so \(C\)~preserves composites. Therefore \(C\)~is a functor. Since \(C\)~is clearly its own inverse, \(C\)~is an isomorphism.\qedhere
\end{enumerate}
\end{proof}
\end{exer}

\begin{exer}[3]\
\begin{enumerate}[itemsep=0pt]
\item[(a)] In~\(\Sets\), the isomorphisms are precisely the bijections.
\item[(b)] In~\(\Mon\), the isomorphisms are precisely the bijective homomorphisms.
\item[(c)] In~\(\Pos\), the isomorphisms are \emph{not} the bijective homomorphisms.
\end{enumerate}
\begin{proof}\
\begin{enumerate}
\item[(a)] A function \(f:A\to B\) has a (two-sided) inverse if and only if it is bijective. Indeed, suppose \(g:B\to A\) is an inverse of~\(f\). If \(a,a'\in A\) and \(f(a)=f(a')\), then
\[a=1_A(a)=(g\after f)(a)=g(f(a))=g(f(a'))=(g\after f)(a')=1_A(a')=a'\]
If \(b\in B\), then \(b=1_B(b)=(f\after g)(b)=f(g(b))\). Conversely, if \(f\)~is bijective, then for each \(b\in B\) we can let~\(g(b)\) be the unique \(a\in A\) with \(f(a)=b\). Then \(g:B\to A\) is clearly an inverse of~\(f\).
\item[(b)] A monoid homomorphism is, in particular, a function, hence an isomorphism is a bijective homomorphism by~(a). Conversely, if \(f:A\to B\) is a bijective homomorphism, then \(f\)~has an inverse \emph{function} \(g:B\to A\) by~(a). If \(b,b'\in B\), then
\[bb'=1_B(b)1_B(b')=(f\after g)(b)(f\after g)(b')=f(g(b))f(g(b'))=f(g(b)g(b'))\]
so
\[g(bb')=g(f(g(b)g(b')))=(g\after f)(g(b)g(b'))=1_A(g(b)g(b'))=g(b)g(b')\]
Therefore \(g\)~is a homomorphism and hence \(f\)~is an isomorphism.
\item[(c)] As in~(b), a poset homomorphism is, in particular, a function, hence an isomorphism is a bijective homomorphism by~(a). However, unlike in~(b), the inverse of a bijective homomorphism need not be a homomorphism. For example, consider a poset consisting of two copies of \(\N=(N,\le)\) with no relations between the copies. Map this poset into~\(\N\) by ``zipping'' the two copies together, sending one to the evens in order, and the other to the odds in order. This mapping is clearly a bijective homomorphism, but its inverse is not since, for example, \(0\le 1\) in the image, but the preimage of~\(0\) is not related to the preimage of~\(1\).\qedhere
\end{enumerate}
\end{proof}
\end{exer}

\begin{exer}[5]
Let \(\C\)~be a category and \(C\in\C\). Let \(U:\C/C\to\C\) ``forget about the base object~\(C\)'' by mapping each object \(f:A\to C\) to its domain~\(A\) and each arrow \(a:A\to B\) to ``itself.''\footnote{An arrow in~\(\C/C\) is understood as a triple \((a,f,f')\) where \(a:A\to B\), \(f:A\to C\), and \(f':B\to C\) are arrows in~\(\C\) with \(f=f'\after a\). So \(U((a,f,f'))=a\).} Then \(U\)~is a functor.

Let \(F:\C/C\to\arr{\C}\) map objects to themselves and each arrow \(a:A\to B\) to the pair~\((a,1_C)\), where \(1_C\)~is the identity arrow for~\(C\) in~\(\C\). Then \(F\)~is a functor, and \(\fdom\after F=U\), where \(\fdom:\arr{\C}\to\C\) is the functor mapping each object \(f:A\to B\) to its domain~\(A\) and each arrow \((g_1,g_2)\) to~\(g_1\).
\end{exer}
\begin{proof}
\(U\)~maps objects to objects and arrows to arrows, and preserves domains and codomains of arrows. Since \(\C/C\) inherits identities and composites from~\(\C\), \(U\)~also preserves identities and composites. Therefore \(U\)~is a functor.

\(F\)~maps objects to objects and arrows to arrows, and preserves domains and codomains of arrows, since if \(a:A\to B\) maps \(f:A\to C\) to \(f':B\to C\) in~\(\C/C\), then \(1_C\after f=f=f'\after a\), hence \((a,1_C)\)~maps~\(f\) to~\(f'\) in~\(\arr{\C}\). Since \(\arr{\C}\)~also inherits identities and composites from~\(\C\), \(F\)~also preserves identities and composites. Therefore \(F\)~is a functor. Clearly \(\fdom\after F=U\).
\end{proof}

\begin{exer}[6]
Let \(\C\)~be a category and \(C\in\C\). Then \(C/\C=\Cop/C\).\footnote{In this exercise, we informally identify an arrow \(f:A\to B\) in~\(\C\) with the corresponding reversed arrow \(\star{f}:\star{B}\to\star{A}\) in~\(\Cop\), and with any corresponding arrows in slices or coslices.}
\end{exer}
\begin{proof}
The arrows out of~\(C\) in~\(\C\) are precisely the arrows into~\(C\) in~\(\Cop\), and the commuting triangles among the former in~\(\C\) are precisely the commuting triangles among the latter in~\(\Cop\). Therefore \(C/\C\)~and~\(\Cop/C\) have the same objects and arrows. They also have the same identities and composites since these are inherited from \(\C\)~and~\(\Cop\), respectively, where they are the same by definition of~\(\Cop\).
\end{proof}
\begin{rmk}
Similarly \(C/\Cop=\C/C\).
\end{rmk}

\begin{exer}[8]
For a (small) category~\(\C\), let \(P(\C)\) consist of the objects from~\(\C\) ordered as follows:
\begin{center}
\(A\le B\) if and only if there exists an arrow \(A\to B\) in~\(\C\)
\end{center}
Then \(P(\C)\)~is a preorder, and \(P:\Cat\to\Pre\) determines a functor with \(P\after C=1_{\Pre}\), where \(C:\Pre\to\Cat\) is the evident inclusion functor.
\end{exer}
\begin{proof}
Reflexivity and transitivity of the order in~\(P(\C)\) follow from the existence of identities and composites in~\(\C\). So \(P\)~maps categories to preorders. For a functor \(F:\C\to\D\), let \(P(F)\)~be the restriction of~\(F\) to objects. If \(A\le B\) in~\(P(\C)\), there exists an arrow \(f:A\to B\) in~\(\C\). But then \(F(f):F(A)\to F(B)\) is an arrow in~\(\D\), so \(F(A)\le F(B)\) in~\(P(\D)\). Therefore \(P(F)\)~is monotone, and hence \(P\)~maps functors to preorder homomorphisms. Since \(P\)~just restricts functors to objects, it preserves domains and codomains, identities, and composites, hence it is a functor. It is obvious that \(P\after C=1_{\Pre}\).
\end{proof}
\begin{rmk}
In general \(C\after P\ne 1_{\Cat}\) because \(P\)~loses information about the arrow structure of categories. Specifically, multiple arrows from one object to another will be represented by a single relation between those objects under~\(P\).
\end{rmk}

\begin{exer}[11]
There exists a functor \(M:\Sets\to\Mon\) mapping each set~\(A\) to the free monoid on~\(A\).
\end{exer}
\begin{proof}
We prove this in two ways.
\begin{enumerate}[itemsep=0pt]
\item[(a)] Let \(M(A)=\star{A}\) and for \(f:A\to B\) define \(M(f):\star{A}\to\star{B}\) by
\[M(f)(a_1\cdots a_k)=f(a_1)\cdots f(a_k)\quad a_1,\ldots,a_k\in A\]
\(M(f)\)~is well defined on~\(\star{A}\) since every element in~\(\star{A}\) can be expressed uniquely as a product of elements of~\(A\), and by construction \(M(f)\)~is a monoid homomorphism extending~\(f\). So \(M\)~maps objects to objects and arrows to arrows. Clearly \(M\)~preserves domains and codomains of arrows and \(M(1_A)=1_{\star{A}}\). If \(g:B\to C\), then
\begin{align*}
M(g\after f)(a_1\cdots a_k)&=(g\after f)(a_1)\cdots(g\after f)(a_k)\\
	&=g(f(a_1))\cdots g(f(a_k))\\
	&=M(g)(f(a_1)\cdots f(a_k))\\
	&=M(g)(M(f)(a_1\cdots a_k))\\
	&=(M(g)\after M(f))(a_1\cdots a_k)
\end{align*}
So \(M(g\after f)=M(g)\after M(f)\) and \(M\)~preserves composites. Therefore \(M\)~is a functor.
\item[(b)] Let \(M(A)\)~be ``the'' free monoid on~\(A\) satisfying the universal mapping property (Propositions 1.9~and~1.10). For \(f:A\to B\to\under{M(B)}\), let \(M(f)\)~be the unique monoid homomorphism from~\(M(A)\) to~\(M(B)\) extending~\(f\). Clearly \(M\)~maps objects to objects and arrows to arrows, and preserves domains and codomains of arrows. Now \(1_{M(A)}\)~extends~\(1_A\), hence we must have \(M(1_A)=1_{M(A)}\). Similarly if \(g:B\to C\to\under{M(C)}\), then \(M(g)\after M(f)\) extends~\(g\after f\), hence we must have \(M(g\after f)=M(g)\after M(f)\). Therefore \(M\)~is a functor.\qedhere
\end{enumerate}
\end{proof}
\begin{rmk}
A homomorphism \(h:M(A)\to B\) is uniquely determined by its action on~\(A\), where this action is \(\under{h}\after i:A\to\under{M(A)}\to\under{B}\). This is trivially true by the universal mapping property since \(h\)~extends~\(\under{h}\after i\) to~\(M(A)\), that is, \(\under{h}\after i=\under{h}\after i\). This is a familiar concept in mathematics (for example, a linear transformation of a vector space is uniquely determined by its action on a basis, etc.).
\end{rmk}

\section*{Chapter~2}
\begin{exer}[1]
In~\(\Sets\), the epis are precisely the surjections. Therefore the isos are precisely the epi-monos.
\end{exer}
\begin{proof}
If \(f:A\to B\) is a surjection, then \(f\)~has a right inverse (AC), hence \(f\)~is a split epi. Conversely, if \(f\)~is not a surjection, there exists \(b\in B\) with \(b\not\in f[A]\). Define \(g:B\to 2\) by
\[g(x)=\begin{cases}
1&\text{if }x=b\\
0&\text{otherwise}
\end{cases}\]
Then \(g\ne0\), but \(g\after f=0\after f\), so \(f\)~is not an epi. Therefore the epis are precisely the surjections.

Now by this result and Proposition~2.2, the epi-monos are precisely the bijections. By Exercise~1.3, the bijections are precisely the isos. Therefore the epi-monos are precisely the isos.
\end{proof}

\begin{exer}[2]
In a poset category, every arrow is an epi-mono since there is at most one arrow between any two objects.
\end{exer}

\begin{exer}[3]
Inverses are unique.
\end{exer}
\begin{proof}
If \(f:A\to B\) and \(g,g':B\to A\) are inverses of~\(f\), then
\[g=g\after 1_B=g\after(f\after g')=(g\after f)\after g'=1_A\after g'=g'\qedhere\]
\end{proof}

\begin{exer}[4]
Let \(f:A\to B\), \(g:B\to C\), and \(h:A\to C\) form a commutative triangle (\(h=g\after f\)):
\begin{diagram}[nohug]
A	&\rTo^f	&B\\
	&\rdTo_h&\dTo>g\\
	&		&C
\end{diagram}
\begin{enumerate}[itemsep=0pt]
\item[(a)] If \(f\)~and~\(g\) are monic [epic, iso], so is~\(h\).
\item[(b)] If \(h\)~is monic, so is~\(f\).
\item[(c)] If \(h\)~is epic, so is~\(g\).
\item[(d)] If \(h\)~is monic, \(g\)~need not be.
\item[(e)] If \(h\)~is epic, \(f\)~need not be.
\end{enumerate}
\begin{proof}\
\begin{enumerate}[itemsep=0pt]
\item[(a)] Suppose \(f\)~and~\(g\) are monic. If \(x,y:D\to A\) and \(h\after x=h\after y\), then
\[g\after(f\after x)=(g\after f)\after x=h\after x=h\after y=(g\after f)\after y=g\after(f\after y)\]
so \(f\after x=f\after y\) since \(g\)~is monic, and \(x=y\) since \(f\)~is monic. Therefore \(h\)~is monic.

Suppose \(f\)~and~\(g\) are epic. If \(i,j:C\to D\) and \(i\after h=j\after h\), then
\[(i\after g)\after f=i\after (g\after f)=i\after h=j\after h=j\after(g\after f)=(j\after g)\after f\]
so \(i\after g=j\after g\) since \(f\)~is epic, and \(i=j\) since \(g\)~is epic. Therefore \(h\)~is epic.\footnote{This also follows from the previous result by duality. If \(f\)~and~\(g\) are epic in~\(\C\), then \(\star{f}\)~and~\(\star{g}\) are monic in~\(\Cop\), so \(\star{h}\)~is monic in~\(\Cop\), so \(h\)~is epic in~\(\C\).}

If \(f\)~and~\(g\) are isos, then \(\inv{h}=\inv{f}\after\inv{g}\), so \(h\)~is an iso.
\item[(b)] If \(f\)~is not monic, choose \(x\ne y\) such that \(f\after x=f\after y\). Then
\[h\after x=(g\after f)\after x=g\after(f\after x)=g\after(f\after y)=(g\after f)\after y=h\after y\]
So \(h\)~is not monic.
\item[(c)] If \(g\)~is not epic, choose \(i\ne j\) such that \(i\after g=j\after g\). Then
\[i\after h=i\after(g\after f)=(i\after g)\after f=(j\after g)\after f=j\after(g\after f)=j\after h\]
So \(h\)~is not epic.\footnote{This also follows from the previous result by duality. If \(h\)~is epic in~\(\C\), then \(\star{h}=\star{f}\after\star{g}\)~is monic in~\(\Cop\), so \(\star{g}\)~is monic in~\(\Cop\), so \(g\)~is epic in~\(\C\).}
\item[(d),(e)] In~\(\Sets\), let \(A=C=1\) and \(B=2\) and let \(f=0_{A\to B}\) and \(g=0_{B\to C}\). Then \(h=0_{A\to C}\) is both monic and epic, but \(g\)~is not monic and \(f\)~is not epic.\qedhere
\end{enumerate}
\end{proof}
\end{exer}

\begin{exer}[5]
For \(f:A\to B\), the following are equivalent:
\begin{enumerate}[itemsep=0pt]
\item[(a)] \(f\)~is an iso.
\item[(b)] \(f\)~is a mono and a split epi.
\item[(c)] \(f\)~is a split mono and an epi.
\item[(d)] \(f\)~is a split mono and a split epi.
\end{enumerate}
\begin{proof}
It is immediate that (a)~\(\implies\)~(d)~\(\implies\)~(b),(c).

For (b)~\(\implies\)~(a), suppose that \(f\)~is monic and \(g:B\to A\) satisfies \(f\after g=1_B\). We claim \(g\)~also satisfies \(g\after f=1_A\), so \(f\)~is an iso. But this follows from
\[f\after(g\after f)=(f\after g)\after f=1_B\after f=f=f\after 1_A\]
since \(f\)~is monic.

For (c)~\(\implies\)~(a), suppose that \(f\)~is epic and \(g:B\to A\) satisfies \(g\after f=1_A\). We claim \(g\)~also satisfies \(f\after g=1_B\), so \(f\)~is an iso. But this follows from
\[(f\after g)\after f=f\after(g\after f)=f\after 1_A=f=1_B\after f\]
since \(f\)~is epic.\footnote{This also follows from the previous result by duality. If \(f\)~is epic and \(g\)~is a left inverse of~\(f\) in~\(\C\), then \(\star{f}\)~is monic and \(\star{g}\)~is a right inverse of~\(\star{f}\) in~\(\Cop\). Therefore \(\star{f}\)~is an iso in~\(\Cop\), so \(f\)~is an iso in~\(\C\).} 
\end{proof}
\end{exer}

\begin{exer}[7]
A retract of a projective object is projective.
\end{exer}
\begin{proof}
Let \(P\)~be projective and \(R\)~be a retract of~\(P\) where \(s:R\to P\), \(r:P\to R\), and \(r\after s=1_R\). Suppose \(f:R\to Y\) and \(e:X\epi Y\). Note \(f\after r:P\to Y\), so by projectivity of~\(P\) there exists \(p:P\to X\) such that \(e\after p=f\after r\):
\begin{diagram}[nohug]
	&						&	&				&X\\
	&						&	&\ruTo(4,2)^p	&\dEpi>e\\
P	&\pile{\rTo^r\\\lTo_s}	&R	&\rTo_f			&Y
\end{diagram}
Now \(p\after s:R\to X\) and
\[e\after(p\after s)=(e\after p)\after s=(f\after r)\after s=f\after(r\after s)=f\after 1_R=f\]
Therefore \(R\)~is projective.
\end{proof}

\begin{exer}[8]
In~\(\Sets\), every set is projective.
\end{exer}
\begin{proof}
If \(f:P\to Y\) and \(g:X\epi Y\), then since \(g\)~is surjective (Exercise~1), \(g\)~has a right inverse \(h:Y\to X\) with \(g\after h=1_Y\). Set \(p=h\after f:P\to X\):
\begin{diagram}[nohug]
	&					&X\\
	&\ruTo^{p=h\after f}&\dEpi<g\uTo>h\\
P	&\rTo_f				&Y
\end{diagram}
Then
\[g\after p=g\after(h\after f)=(g\after h)\after f=1_Y\after f=f\]
Therefore \(P\)~is projective. 
\end{proof}
\begin{rmk}
Projectivity is more interesting in categories of structured sets, where it implies ``freeness'' of structure allowing factoring of outgoing morphisms.
\end{rmk}

\begin{exer}[11]
For a set~\(A\), let \(A\text{-}\Mon\) be the category of \(A\)-monoids \((M,m)\), where \(M\)~is a monoid and \(m:A\to U(M)\), with arrows \(h:(M,m)\to(N,n)\), where \(h:M\to N\) is a monoid homomorphism and \(n=U(h)\after m\).

An initial object in \(A\text{-}\Mon\) is just a free monoid on~\(A\) in~\(\Mon\).
\end{exer}
\begin{proof}
The \(A\)-monoid~\((M,m)\) is initial if and only if for all \(A\)-monoids~\((N,n)\), there is a unique \(A\)-monoid homomorphism \(h:(M,m)\to(N,n)\). This is just to say that \(m:A\to U(M)\) and for all monoids~\(N\) with \(n:A\to U(N)\) there is a unique monoid homomorphism \(h:M\to N\) with \(n=U(h)\after m\). But this is just the universal mapping property for the free monoid on~\(A\) in~\(\Mon\).
\end{proof}

\begin{exer}[12]
For a Boolean algebra~\(B\), Boolean homomorphisms \(p:B\to\cat{2}\) correspond exactly to ultrafilters in~\(B\).
\end{exer}
\begin{proof}
If \(p:B\to\cat{2}\) is a homomorphism, then \(U_p=\inv{p}(1)\) is an ultrafilter in~\(B\). Indeed, \(U_p\ne\emptyset\) since \(p(1)=1\), and \(U_p\ne B\) since \(p(0)=0\ne 1\). If \(a\in U_p\) and \(a\le b\), then \(1=p(a)\le p(b)\), so \(p(b)=1\) and \(b\in U_p\). If \(a\in U_p\) and \(b\in U_p\), then
\[p(a\meet b)=p(a)\meet p(b)=1\meet 1=1\]
so \(a\meet b\in U_p\). Finally, if \(a\not\in U_p\) then \(p(a)=0\), so \(p(\compl a)=\compl p(a)=1\) and \(\compl a\in U_p\). On the other hand, \(p(a)\meet p(\compl a)=0\), so we cannot have \(a\in U_p\) and \(\compl a\in U_p\).

Conversely, if \(U\)~is an ultrafilter in~\(B\) and \(p_U:B\to\cat{2}\) is defined by
\[
p_U(x)=\begin{cases}
1&\text{if }x\in U\\
0&\text{otherwise}
\end{cases}
\]
then \(p\)~is a homomorphism, by reasoning similar to that above. Clearly the maps \(p\mapsto U_p\) and \(U\mapsto p_U\) are mutually inverse.
\end{proof}

\begin{exer}[13]
In any category with binary products,
\[(A\times B)\times C\iso A\times (B\times C)\]
\end{exer}
\begin{proof}
Instead of using the universal property of a ternary product, we use only the universal properties of the binary products. Define
\[f=\pair{p_A\after p_{A\times B}}{\pair{p_B\after p_{A\times B}}{p_C}}:(A\times B)\times C\to A\times(B\times C)\]
where the \(p\)'s are the obvious projections, and
\[g=\pair{\pair{q_A}{q_B\after q_{B\times C}}}{q_C\after q_{B\times C}}:A\times(B\times C)\to(A\times B)\times C\]
where the \(q\)'s are the obvious projections. Then \(f\) and~\(g\) are inverses, so they are isomorphisms. For example,
\begin{align*}
f\after g&=\pair{p_A\after p_{A\times B}}{\pair{p_B\after p_{A\times B}}{p_C}}\after g\\
	&=\pair{p_A\after p_{A\times B}\after g}{\pair{p_B\after p_{A\times B}\after g}{p_C\after g}}\\
	&=\pair{p_A\after \pair{q_A}{q_B\after q_{B\times C}}}{\pair{p_B\after \pair{q_A}{q_B\after q_{B\times C}}}{q_C\after q_{B\times C}}}\\
	&=\pair{q_A}{\pair{q_B\after q_{B\times C}}{q_C\after q_{B\times C}}}\\
	&=\pair{q_A}{q_{B\times C}}\\
	&=1_{A\times(B\times C)}\qedhere
\end{align*}
\end{proof}

\begin{exer}[15]
For a category~\(\C\) and objects \(A,B\in\C\), let \(\C_{A,B}\)~be the category with objects \((X,x_1,x_2)\), where \(x_1:X\to A\) and \(x_2:X\to B\) in~\(\C\), and with arrows \(f:(X,x_1,x_2)\to(Y,y_1,y_2)\), where \(f:X\to Y\) and \(x_i=y_i\after f\) in~\(\C\).

A terminal object in~\(\C_{A,B}\) is just a product of \(A\)~and~\(B\) in~\(\C\).
\end{exer}
\begin{proof}
Object \((P,p_1,p_2)\) in~\(\C_{A,B}\) is terminal if and only if for all objects \((X,x_1,x_2)\) there is a unique \(p:(X,x_1,x_2)\to(P,p_1,p_2)\). This is just to say that \(p_1:P\to A\), \(p_2:P\to B\), and for all objects \(X\in\C\) with arrows \(x_1:X\to A\) and \(x_2:X\to B\) there is a unique \(p:X\to P\) with \(x_i=p_i\after p\). But this is just the universal mapping property for the product~\(A\times B\) in~\(\C\).
\end{proof}
\begin{rmk}
The objects in~\(\C_{A,B}\) are just pairs of ``generalized elements'' of \(A\)~and~\(B\) in~\(\C\). A terminal object in~\(\C_{A,B}\) has a unique ``generalized element'' for every such pair, hence it is just the product~\(A\times B\).
\end{rmk}

\begin{exer}[16]
Let \(\Types\)~be the category of types in the \(\lambda\)-calculus. Then the product functor \(\times:\Types\times\Types\to\Types\) maps objects \(A\)~and~\(B\) to~\(A\times B\) and arrows \(f:A\to B\) and \(g:A'\to B'\) to \(f\times g:A\times A'\to B\times B'\) where
\[f\times g=\lambda c.\pair{f(\fst(c))}{g(\snd(c))}\]
For any fixed type~\(A\), there is a functor~\(A\to(-)\) on~\(\Types\) taking each type~\(X\) to the type~\(A\to X\).
\end{exer}
\begin{proof}
We know \(\Types\)~has products, so the product functor is defined on~\(\Types\). For \(f:A\to B\) and \(g:A'\to B'\), if
\[A\lTo^{p_1}A\times A'\rTo^{p_2}A'\]
where \(p_1=\lambda z.\fst(z)\) and \(p_2=\lambda z.\snd(z)\), then
\begin{align*}
f\times g&=\pair{f\after p_1}{g\after p_2}\\
	&=\lambda c.\pair{f(p_1c)}{g(p_2c)}\\
	&=\lambda c.\pair{f(\fst(c))}{g(\snd(c))}
\end{align*}
Fix a type~\(A\). Let \(A\to(-)\)~map each type~\(X\) to the type~\(A\to X\) and map each function \(f:X\to Y\) to the function \(\overline{f}:(A\to X)\to(A\to Y)\) given by \(\overline{f}=\lambda g.f\after g\), where \(f\after g=\lambda x.f(gx)\). We claim this mapping is a functor.

Indeed, this mapping clearly maps objects to objects and arrows to arrows and it preserves domains and codomains of arrows. It also clearly preserves identities. If \(g:Y\to Z\), then
\begin{align*}
\overline{g\after f}&=\lambda h.(g\after f)\after h\\
	&=\lambda h.g\after(f\after h)\\
	&=\lambda h.\overline{g}(\overline{f}h)\\
	&=\overline{g}\after\overline{f}
\end{align*}
So the mapping also preserves composites.\footnote{Note preservation of identities and composites relies on \(\beta\eta\)-equivalence for \emph{equality} of the functions involved.} Therefore it is a functor.
\end{proof}
\begin{rmk}
This result shows that in functional programming languages such as Haskell, functions of a fixed input type are ``functorial'' types. This implies that functions on arbitrary types can be lifted to operate on such functions through composition.
\end{rmk}

\begin{exer}[17]
In any category~\(\C\) with products, define the \emph{graph} of an arrow \(f:A\to B\) by
\[\Gamma(f)=\pair{1_A}{f}:A\mono A\times B\]
Then \(\Gamma(f)\)~is a mono for every arrow~\(f\).

In~\(\Sets\), \(\Gamma\)~determines a functor \(G:\Sets\to\Rel\) mapping sets to themselves and functions to their graphs.
\end{exer}
\begin{proof}
To see that \(\Gamma(f)\)~is a mono, suppose \(x,y:X\to A\) and \(\Gamma(f)\after x=\Gamma(f)\after y\). By the universal mapping property of~\(A\times B\),
\[\Gamma(f)\after x=\pair{1_A}{f}\after x=\pair{1_A\after x}{f\after x}=\pair{x}{f\after x}\]
Similarly \(\Gamma(f)\after y=\pair{y}{f\after y}\). Therefore \(\pair{x}{f\after x}=\pair{y}{f\after y}\), so \(x=y\).

Define \(G:\Sets\to\Rel\) by \(G(A)=A\) and \(G(f:A\to B)=\Gamma(f)[A]\subseteq A\times B\). Then clearly \(G\)~is just the map from Exercise~1.1(b), which is a functor.
\end{proof}

\begin{rmk}
Recall that a monoid homomorphism \(h:M(A)\to B\) is uniquely determined by its action on~\(A\). For inclusion \(i:A\to\under{M(A)}\) and homomorphisms \(j,k:M(A)\to B\), this implies that if \(\under{j}\after i=\under{k}\after i\) then \(j=k\). So while \(i\)~is not an epi in~\(\Sets\) (it is not a surjection) and is not even an arrow in~\(\Mon\) (it is not a homomorphism), it is like an epi if we blur the line between \(\Sets\)~and~\(\Mon\). It is ``structurally surjective'' in the sense that once a homomorphic structure is determined on~\(A\), it is determined on~\(M(A)\).\footnote{Compare with Example~2.5.}
\end{rmk}

\section*{Chapter~3}
\begin{exer}[1]
Let \(\C\)~be a (locally small) category. Then
\[A\rTo^{c_1}C\lTo^{c_2}B\]
is a coproduct if and only if for all objects~\(Z\) the function
\begin{align*}
\Hom(C,Z)&\to\Hom(A,Z)\times\Hom(B,Z)\\
f&\mapsto\pair{f\after c_1}{f\after c_2}
\end{align*}
is an iso.
\end{exer}
\begin{proof}
By duality, the given diagram is a coproduct in~\(\C\) if and only if
\[\star{A}\lTo^{\star{c_1}}\star{C}\rTo^{\star{c_2}}\star{B}\]
is a product in~\(\Cop\). We know this diagram is a product in~\(\Cop\) if and only if for all objects~\(\star{Z}\), the function
\begin{align*}
\Hom_{\Cop}(\star{Z},\star{C})&\to\Hom_{\Cop}(\star{Z},\star{A})\times\Hom_{\Cop}(\star{Z},\star{B})\\
\star{f}&\mapsto\pair{\star{c_1}\after\star{f}}{\star{c_2}\after\star{f}}
\end{align*}
is an iso (Proposition~2.20). But by definition of~\(\Cop\),
\[\Hom_{\Cop}(\star{Z},\star{X})=\Hom_{\C}(X,Z)\]
for all objects \(X\in\C\) and \(\star{c_i}\after\star{f}=\star{(f\after c_i)}\) for all arrows \(f\in\C\), so the functions are the same.
\end{proof}

\begin{exer}[2]
The free monoid functor preserves coproducts, that is,
\[M(A+B)\iso M(A)+M(B)\]
\end{exer}
\begin{proof}
By the universal property of~\(M(A)\), let \(i_1:M(A)\to M(A+B)\) extend the inclusion \(A\to A+B\to\under{M(A+B)}\).\footnote{The inclusion \(A\to A+B\) is from the coproduct construction, and the inclusion \(A+B\to\under{M(A+B)}\) is from the free monoid construction. Observe \(A\to A+B\) is lifted to~\(i_1\) by the free monoid functor.} Similarly let \(i_2:M(B)\to M(A+B)\) extend the inclusion \(B\to A+B\to\under{M(A+B)}\). We claim
\[M(A)\rTo^{i_1}M(A+B)\lTo^{i_2}M(B)\]
is a coproduct of \(M(A)\)~and~\(M(B)\), from which the desired result follows by uniqueness of the coproduct (Proposition~3.12).

Given \(x:M(A)\to N\) and \(y:M(B)\to N\), let \(\under{x}_A:A\to\under{N}\) be the composite of the inclusion \(A\to\under{M(A)}\) with \(\under{x}:\under{M(A)}\to\under{N}\), and similarly let \(\under{y}_B:B\to\under{N}\) be the composite of the inclusion \(B\to\under{M(B)}\) with \(\under{y}:\under{M(B)}\to\under{N}\). By the universal property of~\(A+B\), there is a unique copairing~\(\copair{\under{x}_A}{\under{y}_B}:A+B\to\under{N}\), and by the universal property of~\(M(A+B)\) this copairing extends uniquely to a homomorphism \(z:M(A+B)\to N\).

It follows from the copairing and the universal property of~\(M(A)\) that \(z\after i_1=x\) since both \(z\after i_1\)~and~\(x\) extend~\(\under{x}_A\) to~\(M(A)\), and similarly \(z\after i_2=y\). Moreover it follows from uniqueness of the copairing and the extension that \(z\)~uniquely satisfies these equations. Therefore \(z\)~is a copairing~\(\copair{x}{y}\), and \(M(A+B)\)~is a coproduct of \(M(A)\)~and~\(M(B)\) as claimed.
\end{proof}

\begin{exer}[5]
Let \(\C\)~be the category of proofs in a natural deduction system with disjunction introduction rules
\[\begin{array}{c}
\varphi\\
\hline
\varphi\lor\psi
\end{array}
\qquad
\begin{array}{c}
\psi\\
\hline
\varphi\lor\psi
\end{array}\]
and disjunction elimination rule
\[\begin{array}{ccc}
&[\varphi]&[\psi]\\
&\vdots&\vdots\\
\varphi\lor\psi&\vartheta&\vartheta\\
\hline
&\vartheta&
\end{array}\]
Note the introduction rules give proofs \(i_1:\varphi\to\varphi\lor\psi\) and \(i_2:\psi\to\varphi\lor\psi\), and the elimination rule gives a proof \(\copair{p}{q}:\varphi\lor\psi\to\vartheta\) from proofs \(p:\varphi\to\vartheta\) and \(q:\psi\to\vartheta\).

For any proofs \(p:\varphi\to\vartheta\), \(q:\psi\to\vartheta\), and \(r:\varphi\lor\psi\to\vartheta\), identify proofs under the equations
\[\copair{p}{q}\after i_1=p\qquad \copair{p}{q}\after i_2=q\qquad [r\after i_1,r\after i_2]=r\]
to disregard unnecessary introduction and elimination of disjunction.

Then \(\C\)~has coproducts, and in fact \(\varphi+\psi=\varphi\lor\psi\).
\end{exer}
\begin{proof}
To see that
\[\varphi\rTo^{i_1}\varphi\lor\psi\lTo^{i_2}\psi\]
is a coproduct, suppose \(p:\varphi\to\vartheta\) and \(q:\psi\to\vartheta\) are proofs. Let \(r:\varphi\lor\psi\to\vartheta\) be the proof given by application of the elimination rule to \(p\)~and~\(q\). Then by the first two identification rules, \(r\after i_1=p\) and \(r\after i_2=q\). If \(s:\varphi\lor\psi\to\vartheta\) also satisfies these properties, then by the second identification rule
\[s=[s\after i_1,s\after i_2]=[p,q]=[r\after i_1,r\after i_2]=r\]
Therefore \(r\)~is unique, and so \(\varphi\lor\psi\)~is a coproduct.
\end{proof}
\begin{rmk}
Dually, it can be shown that the category of proofs in a system with conjunction elimination rules
\[\begin{array}{c}
\varphi\land\psi\\
\hline
\varphi
\end{array}
\qquad
\begin{array}{c}
\varphi\land\psi\\
\hline
\psi
\end{array}\]
determining proofs \(p_1:\varphi\land\psi\to\varphi\) and \(p_2:\varphi\land\psi\to\psi\), together with conjunction introduction rule
\[\begin{array}{ccc}
&[\vartheta]&[\vartheta]\\
&\vdots&\vdots\\
\vartheta&\varphi&\psi\\
\hline
&\varphi\land\psi&
\end{array}\]
determining a proof \(\pair{p}{q}:\vartheta\to\varphi\land\psi\) from proofs \(p:\vartheta\to\varphi\) and \(q:\vartheta\to\psi\), all under identification rules
\[p_1\after\pair{p}{q}=p\qquad p_2\after\pair{p}{q}=q\qquad\pair{p_1\after r}{p_2\after r}=r\]
for arbitrary proofs \(p:\vartheta\to\varphi\), \(q:\vartheta\to\psi\), and \(r:\vartheta\to\varphi\land\psi\), has products, and in fact \(\varphi\times\psi=\varphi\land\psi\).
\end{rmk}

\begin{exer}[6]
The category~\(\Mon\) has all equalizers.
\end{exer}
\begin{proof}
Given any monoid homomorphisms \(f,g:A\to B\), define the set
\[E=\{\,x\in A\mid f(x)=g(x)\,\}\]
We claim \(E\)~is a submonoid of~\(A\). Indeed, \(u_A\in E\) since \(f(u_A)=u_B=g(u_A)\), and if \(x,y\in E\) then \(xy\in E\) since
\[f(xy)=f(x)f(y)=g(x)g(y)=g(xy)\]
Let \(i:E\to A\) be the inclusion homomorphism. It follows that \(i\)~is an equalizer of \(f\)~and~\(g\), by the same argument used in~\(\Sets\).
\end{proof}

\begin{exer}[7]
Let \(\C\)~be a category with coproducts. If \(P\)~and~\(Q\) are projective, then \(P+Q\)~is projective.
\end{exer}
\begin{proof}
Suppose
\[P\rTo^{i_1}P+Q\lTo^{i_2}Q\]
is a coproduct diagram. If \(f:P+Q\to X\) and \(e:E\epi X\), then by projectivity of \(P\)~and~\(Q\) there exist arrows \(\overline{f\after i_1}:P\to E\) and \(\overline{f\after i_2}:Q\to E\) satisfying equations \(e\after\overline{f\after i_k}=f\after i_k\). Define \(\overline{f}=\copair{\overline{f\after i_1}}{\overline{f\after i_2}}\). Then
\[e\after\overline{f}
=e\after\copair{\overline{f\after i_1}}{\overline{f\after i_2}}
=\copair{e\after\overline{f\after i_1}}{e\after\overline{f\after i_2}}
=\copair{f\after i_1}{f\after i_2}
=f\]
Therefore \(P+Q\)~is projective.
\end{proof}

\begin{exer}[8]
An object~\(Q\) is \emph{injective} in a category~\(\C\) if \(\star{Q}\)~is projective in~\(\Cop\), that is, if for all arrows \(f:X\to Q\) and monos \(m:X\mono A\), there exists \(\overline{f}:A\to Q\) such that \(\overline{f}\after m=f\):
\begin{diagram}[nohug]
X		&\rTo^f					&Q\\
\dMono<m&\ruTo_{\overline{f}}	&\\
A		&						&
\end{diagram}

In~\(\Pos\), the empty poset is not injective, but the singleton poset is injective.
\end{exer}
\begin{proof}
For the empty poset, consider any nonempty~\(A\).
\end{proof}

\begin{exer}[11]
The category~\(\Sets\) has all coequalizers.
\end{exer}
\begin{proof}
Given any functions \(f,g:A\to B\), let \(\eq\)~be the equivalence relation on~\(B\) generated by pairs \(f(x)\eq g(x)\) for all \(x\in A\).\footnote{This is defined as the intersection of all equivalence relations on~\(B\) containing all such pairs, which is the smallest equivalence relation containing all such pairs. Note this intersection is well defined since \(B\times B\) is an equivalence relation on~\(B\) containing all such pairs.} Let \(C\)~be the quotient~\(B/\eq\). We claim the projection \(\pi:B\to C\) given by \(y\mapsto[y]\) is a coequalizer of \(f\)~and~\(g\).

Clearly \(\pi\after f=\pi\after g\) since for \(x\in A\), \(f(x)\eq g(x)\), hence
\[\pi(f(x))=[f(x)]=[g(x)]=\pi(g(x))\]
Suppose \(h:B\to D\) satisfies \(h\after f=h\after g\). Let \(\eq_h\)~be the equivalence relation on~\(B\) defined by
\[y\eq_h z\iff h(y)=h(z)\]
Note \(f(x)\eq_h g(x)\) for all \(x\in A\) since \(h\after f=h\after g\). This implies \({\eq}\subseteq{\eq_h}\), so if \(y\eq z\) then \(h(y)=h(z)\). In other words, \(h\)~respects~\(\eq\). Define \(\overline{h}:C\to D\) by \([y]\mapsto h(y)\). Then \(\overline{h}\)~is well defined since \(h\)~respects~\(\eq\), and \(\overline{h}\after\pi=h\). Moreover, \(\overline{h}\)~is unique since \(\pi\)~is epic (surjective).

Therefore \(\pi\)~is a coequalizer of \(f\)~and~\(g\).
\end{proof}

\begin{exer}[14]
In the category~\(\Sets\):
\begin{enumerate}[itemsep=0pt]
\item[(a)] If \(f:A\to B\) and
\[A\lTo^{p_1}A\times A\rTo^{p_2}A\]
then the equalizer of \(f\after p_1\)~and~\(f\after p_2\) is an equivalence relation on~\(A\), called the \emph{kernel} of~\(f\).
\item[(b)] If \(R\)~is an equivalence relation on~\(A\) and \(\pi:A\to A/R\) is the projection \(x\mapsto[x]\), then \(\ker\pi=R\).
\item[(c)] If \(R\)~is a binary relation on~\(A\) and \(\gen{R}\)~is the equivalence relation on~\(A\) generated by~\(R\), then the projection \(\pi:A\to A/\gen{R}\) is a coequalizer of the projections \(p_1,p_2:R\to A\).
\item[(d)] If \(R\)~is a binary relation on~\(A\), then \(\gen{R}\)~is just the kernel of the coequalizer of the projections \(p_1,p_2:R\to A\).
\end{enumerate}
\end{exer}
\begin{proof}\
\begin{enumerate}[itemsep=0pt]
\item[(a)] We know (Example~3.15) that the equalizer is just (the inclusion of)
\begin{align*}
E&=\{\,(x,y)\mid f\after p_1(x,y)=f\after p_2(x,y)\,\}\\
	&=\{\,(x,y)\mid f(x)=f(y)\,\}\subseteq A\times A
\end{align*}
It is immediate that \(E\)~is an equivalence relation on~\(A\).
\item[(b)] We have
\begin{align*}
(x,y)\in\ker\pi&\iff\pi(x)=\pi(y)\\
	&\iff[x]=[y]\\
	&\iff(x,y)\in R
\end{align*}
\item[(c)] By the proof of Exercise~11 with \(f=p_1\) and \(g=p_2\).
\item[(d)] By part~(b) we know \(\gen{R}=\ker\pi\) where \(\pi:A\to A/\gen{R}\) is the projection \(x\mapsto[x]\), and by part~(c) we know \(\pi\)~is a coequalizer of the projections \(p_1,p_2:R\to A\).\qedhere
\end{enumerate}
\end{proof}
\begin{rmk}
The kernel of a function is just the set of pairs of elements identified or equated by the function. For projecton under an equivalence relation, this is obviously just the relation itself. If we want to identify elements under an \emph{arbitrary} relation (using a quotient construction), then we must also identify elements under the equivalence relation generated by that relation.
\end{rmk}

\section*{Chapter~4}
\begin{rmk}
For a group in a category, the characterizing ``equations''
\begin{gather*}
m(m(x,y),z)=m(x,m(y,z))\\
m(x,u)=x=m(u,x)\\
m(x,ix)=u=m(ix,x)
\end{gather*}
must be understood to \emph{mean} that the following diagrams commute:
\begin{diagram}
(Z\times Z)\times Z					&		&\rTo^{\iso}	&		&Z\times(Z\times Z)\\
\dTo<{(m\after(x\times y))\times z}	&		&				&		&\dTo>{x\times(m\after(y\times z))}\\
G\times G							&\rTo_m	&G				&\lTo_m	&G\times G
\end{diagram}
\begin{diagram}
Z\times 1		&\lTo^{\pair{1_Z}{!}}_{\iso}&Z		&\rTo^{\pair{!}{1_Z}}_{\iso}	&1\times Z\\
\dTo<{x\times u}&							&\dTo<x	&								&\dTo>{u\times x}\\
G\times G		&\rTo_m						&G		&\lTo_m							&G\times G
\end{diagram}
\begin{diagram}
Z\times Z			&\lTo^{\Delta}	&Z			&\rTo^{\Delta}	&Z\times Z\\
\dTo<{x\times ix}	&				&\dTo<{u!}	&				&\dTo>{ix\times x}\\
G\times G			&\rTo_m			&G			&\lTo_m			&G\times G
\end{diagram}
Note that a ``pair'' like \((x,y)\) in an ``equation'' is interpreted as a product arrow \(x\times y\) in the corresponding diagram.

The defining diagrams for~\(G\) commute if and only if the above diagrams commute for all \(x,y,z:Z\to G\). Indeed, for the forward direction, the above diagrams can be factored through the diagrams for~\(G\); for the reverse direction, taking \(x=y=z=1_G\) in the above diagrams yields the diagrams for~\(G\).
\end{rmk}

\begin{rmk}
The homomorphism theorem for groups (Theorem~4.10) shows that for a group homomorphism \(h:G\to H\), \(\ker h\)~is universal among the normal subgroups of~\(G\) factorization through which preserves~\(h\). Equivalently, \(G/\ker h\) is universal among quotients through which \(h\)~is preserved.

In detail, \(K=\ker h\) is a normal subgroup of~\(G\) making the following diagram commute:
\begin{diagram}[nohug]
G			&\rTo^h					&H\\
\dTo<{\pi_K}&\ruTo>{\overline{h_K}}	&\\
G/K			&						&
\end{diagram}
Given any normal subgroup \(N\)~of~\(G\) making the following diagram commute:
\begin{diagram}[nohug]
G			&\rTo^h					&H\\
\dTo<{\pi_N}&\ruTo>{\overline{h_N}}	&\\
G/N			&						&
\end{diagram}
there exists a unique homomorphism \(\pi_{K/N}:G/N\to G/K\) making the following diagram commute:
\begin{diagram}[nohug]
G/N				&\rTo^{\overline{h_N}}	&H\\
\dTo<{\pi_{K/N}}&\ruTo>{\overline{h_K}}	&\\
G/K				&						&
\end{diagram}
Indeed, \(\pi_{K/N}([x]_N)=[x]_K\) is a well defined homomorphism since \(N\subseteq K\), and
\[(\overline{h_K}\after\pi_{K/N})([x]_N)=\overline{h_K}(\pi_{K/N}([x]_N))=\overline{h_K}([x]_K)=h(x)=\overline{h_N}([x]_N)\]
Also \(\pi_{K/N}\)~is unique since \(\overline{h_K}\)~is injective. In other words, \(\overline{h_N}\)~factors uniquely through~\(\overline{h_K}\).

Intuitively, as \(N\)~ranges from \(1\)~to~\(K\), \(G/N\)~``collapses'' more and more of the structure of~\(G\) while still preserving~\(h\). Since \(G/K\)~is the ``smallest'' with this property, it is always possible to collapse from~\(G/N\) to \(G/K\) and still preserve~\(h\).

Observe \(\pi_{K/N}\)~is surjective and \(\ker\pi_{K/N}=K/N\), from which it follows that
\[(G/N)/(K/N)\iso G/K\]
This is just the third isomorphism theorem for groups.
\end{rmk}

\begin{exer}[1]
Let \(G\)~be a group. A categorical congruence~\(\eq\) on~\(G\) (viewed as a category\footnote{Recall a group is a category with only one object in which every arrow is an iso.}) is the same thing as an equivalence relation on~\(G\) determined by a normal subgroup \(N\subseteq G\). Moreover, \(G/{\eq}=G/N\).
\end{exer}
\begin{proof}
If \(\eq\)~is a categorical congruence on~\(G\), then \(\eq\)~determines an equivalence relation on the arrows of~\(G\), which are just the elements of~\(G\). Let \(N=[1]\) be the equivalence class of the identity \(1\in G\). For \(x,y\in G\), observe
\begin{align*}
x\eq y&\iff xy^{-1}\eq yy^{-1}=1&&\text{by closure of~\(\eq\)}\\
	&\iff xy^{-1}\in [1]=N
\end{align*}
Now \(1\in N\), and if \(x,y\in N\) then \(x\eq y\), so \(xy^{-1}\in N\). Hence \(N\)~is a subgroup of~\(G\). Moreover, if \(x\in N\) and \(y\in G\), then again by closure
\[yxy^{-1}\eq y1y^{-1}=yy^{-1}=1\]
so \(yxy^{-1}\in N\). Hence \(N\)~is normal. The above biconditional shows that \(\eq\)~is just the equivalence relation determined by~\(N\).

Conversely, if \(N\)~is a normal subgroup of~\(G\), then the equivalence relation defined by
\[x\eq y\iff xy^{-1}\in N\]
is a categorical congruence. Indeed, if \(x\eq y\) then \(x\)~is trivially parallel to~\(y\) since all arrows are parallel (there being only one object). If \(x\eq y\), then \(xy^{-1}\in N\), so for \(w,z\in G\),
\[wxz(wyz)^{-1}=wxzz^{-1}y^{-1}w^{-1}=w(xy^{-1})w^{-1}\in N\]
since \(N\)~is normal, so \(wxz\eq wyz\). Hence \(\eq\)~is closed under composition.

Now for congruence~\(\eq\), \(G/{\eq}\)~consists of one object and arrows which are the equivalence classes of the arrows in~\(G\) under~\(\eq\), composed by \([x][y]=[xy]\). This arrow structure matches the element structure of~\(G/N\), where \(N\)~is the normal subgroup of~\(G\) corresponding to~\(\eq\). Therefore \(G/{\eq}=G/N\).
\end{proof}
\begin{rmk}
This exercise shows that the homomorphism theorem for categories (Theorem~4.13) is in fact a generalization of the homomorphism theorem for groups (Theorem~4.10).
\end{rmk}

\begin{exer}[3]
If \(G\)~is an abelian group, then \(G\)~is a group in the category of groups.
\end{exer}
\begin{proof}
Since \(G\)~is a group, \(G\)~is an object in the category. Define \(m:G\times G\to G\) by \(m(x,y)=xy\). Then for \((x,y),(x',y')\in G\times G\),
\begin{align*}
m((x,y)(x',y'))&=m(xx',yy')&&\text{since }(x,y)(x',y')=(xx',yy')\\
	&=xx'yy'&&\\
	&=xyx'y'&&\text{since \(G\)~is abelian}\\
	&=m(x,y)m(x',y')&&
\end{align*}
So \(m\)~is a homomorphism, that is, an arrow in the category. Define \(u:1\to G\) by \(u(u)=u\) and \(i:G\to G\) by \(i(x)=x^{-1}\). Clearly \(u\)~is a homomorphism. If \(x,y\in G\), then
\[i(xy)=(xy)^{-1}=y^{-1}x^{-1}=x^{-1}y^{-1}=i(x)i(y)\]
since \(G\)~is abelian. So \(i\)~is a homomorphism.

Now for \(x,y,z\in G\), we have
\[m(m(x,y),z)=m(xy,z)=(xy)z=x(yz)=m(x,yz)=m(x,m(y,z))\]
and
\[m(x,u)=xu=x=ux=m(u,x)\]
and
\[m(x,i(x))=m(x,x^{-1})=xx^{-1}=u=x^{-1}x=m(x^{-1},x)=m(i(x),x)\]
So \(m\)~is associative, \(u\)~is a unit for~\(m\), and \(i\)~is an inverse for~\(m\), for elements of~\(G\). It follows that this is also true for generalized elements of~\(G\), by definition of the homomorphism composition and pairing operations in the category. Therefore \(G\)~is a group in the category.
\end{proof}

\begin{exer}[7]
Let \(\eq\)~be a congruence in~\(\C\). If \(f,f':A\to B\), \(g,g':B\to C\), \(f\eq f'\), and \(g\eq g'\), then \(gf\eq g'f'\).
\end{exer}
\begin{proof}
By two applications of closure, we have
\[gf\eq gf'\eq g'f'\qedhere\]
\end{proof}
\begin{rmk}
Together with the fact that congruent arrows are parallel, this exercise shows that composition in the congruence category~\(\C^{\eq}\) is well defined.
\end{rmk}

\begin{exer}[8]
Let \(F,G:\C\to\D\) be functors such that \(F(X)=G(X)\) for all objects \(X\in\C\). Define a relation~\(\eq\) on the arrows of~\(\D\) as follows:
\begin{quote}
\(f\eq g\) \(\iff\) \(f\)~and~\(g\) are parallel and for all functors \(H:\D\to\E\) with \(HF=HG\), \(H(f)=H(g)\).
\end{quote}
Then \(\eq\)~is a congruence on~\(\D\), and \(\D/{\eq}\)~is a coequalizer of \(F\)~and~\(G\).
\end{exer}
\begin{proof}
It is immediate that \(\eq\)~is an equivalence relation on parallel arrows. If \(f,g:A\to B\), \(a:X\to A\), \(b:B\to Y\), and \(f\eq g\), we claim \(bfa\eq bga\). Indeed, \(bfa\)~and~\(bga\) are parallel since \(f\)~and~\(g\) are parallel, and if \(H:\D\to\E\) is a functor with \(HF=HG\), then
\[H(bfa)=H(b)H(f)H(a)=H(b)H(g)H(a)=H(bga)\]
So \(\eq\)~is closed under composition, and hence is a congruence.

Now if \(f\in\C\), then \(F(f)\)~and~\(G(f)\) are parallel since \(F\)~and~\(G\) agree on objects, and if \(H:\D\to\E\) with \(HF=HG\), then
\[H(F(f))=HF(f)=HG(f)=H(G(f))\]
Therefore \(F(f)\eq G(f)\) for all \(f\in\C\).\footnote{In fact, \(\eq\)~is the congruence generated by pairs \(F(f)\eq G(f)\) for all arrows \(f\in\C\). Compare with Exercise~3.13.}

Let \(P:\D\to\D/{\eq}\) be the projection functor \(f\mapsto[f]\). Then \(PF\)~and~\(PG\) agree on objects since \(F\)~and~\(G\) do, and \(PF\)~and~\(PG\) agree on arrows since \(F(f)\eq G(f)\) for all \(f\in\C\). Therefore \(PF=PG\). If \(H:\D\to\E\) is any functor with \(HF=HG\), then \(H\)~respects the congruence by definition of the congruence, so the functor \(\overline{H}:\D/{\eq}\to\E\) given by \([f]\mapsto H(f)\) is well defined with \(\overline{H}P=H\). Moreover, \(\overline{H}\)~is unique in this regard since \(P\)~is epic.
\end{proof}

\section*{Chapter~5}
\begin{rmk}
The pullback functor \(\Cop\to\Cat\) (Proposition 5.10) is closely related to the slice functor \(\C/(-):\C\to\Cat\) (Section 1.6). They both map an object~\(C\) to the slice category~\(\C/C\). The former maps an arrow to a functor taking the ``inverse image'' (pullback) under that arrow, while the latter maps an arrow to a functor taking the image (composite) under that arrow.
\end{rmk}

\begin{exer}[1]
Let \(\C\)~be a category and \(X\in\C\). A pullback in~\(\C\) over~\(X\) is just a product in~\(\C/X\).
\end{exer}
\begin{proof}
This follows directly from the universal mapping properties. Alternately, we know (Example~5.20) that a pullback is a limit of a diagram of this type:
\begin{diagram}
	&		&\obj\\
	&		&\dTo\\
\obj&\rTo	&\obj
\end{diagram}
Similarly (Example~5.17), a product is a limit of a diagram of this type:
\[\obj\qquad\obj\]
A diagram of the former type in~\(\C\) over~\(X\) is just a diagram of the latter type in~\(\C/X\), so the limits coincide.
\end{proof}

\begin{exer}[3]
A pullback of a mono is a mono.
\end{exer}
\begin{proof}
Suppose \(m:M\mono A\) is a mono and \(m':M'\to A'\) is a pullback of~\(m\) along \(f:A'\to A\). Further suppose \(x,y:X\to M'\) with \(m'x=m'y\):
\begin{diagram}
X	&\pile{\rTo^x\\\rTo_y}	&M'			&\rTo^{f'}	&M\\
	&\rdTo					&\dTo<{m'}	&			&\dMono>m\\
	&						&A'			&\rTo_f		&A
\end{diagram}
Then
\[m(f'x)=(mf')x=(fm')x=f(m'x)=f(m'y)=(fm')y=(mf')y=m(f'y)\]
It follows that \(f'x=f'y\) since \(m\)~is monic. Set \(g=m'x=m'y\) and \(h=f'x=f'y\). Since \(M'\)~is a pullback, there is a unique \(z:X\to M'\) with \(m'z=g\) and \(f'z=h\), and since \(x\)~and~\(y\) both satisfy these equations, it follows that \(x=y\). Therefore \(m'\)~is monic as desired.
\end{proof}

\begin{exer}[4]
Let \(\C\)~be a category, \(A\in\C\), and \(M,N\in\Sub_{\C}(A)\). Then
\[M\subseteq N\iff\forall z:Z\to A\ (z\in_A M\implies z\in_A N)\]
\end{exer}
\begin{proof}
If \(M\subseteq N\), let \(i:M\to N\) satisfy \(m=ni\). For \(z:Z\to A\) with \(z\in_A M\), let \(\overline{z}:Z\to M\) satisfy \(z=m\overline{z}\):
\begin{diagram}[nohug]
Z	&\rTo^{\overline{z}}&M			&\rTo^i		&N\\
	&\rdTo_z			&\dMono>m	&\ldMono>n	&\\
	&					&A			&			&
\end{diagram}
Then
\[z=m\overline{z}=(ni)\overline{z}=n(i\overline{z})\]
So \(i\overline{z}:Z\to N\) witnesses \(z\in_A N\).

Conversely, if \(z\in_A M\) implies \(z\in_A N\), then since \(m\in_A M\) trivially (\(m=m1_A\)), we have \(m\in_A N\). This means there is \(i:M\to N\) with \(m=ni\), so \(M\subseteq N\).
\end{proof}

\begin{exer}[5]
Let \(\C\)~be a category, \(A\in\C\), and \(M,N\in\Sub_{\C}(A)\). Then
\[M\equiv N\iff\forall z:Z\to A\ (z\in_A M\iff z\in_A N)\]
\end{exer}
\begin{proof}
Immediate from Exercise~4.
\end{proof}

\begin{rmk}
The previous two results justify the abuse of subset notation for subobjects through the abuse of set membership notation for generalized elements. 
\end{rmk}

\begin{exer}[6]
Let \(\C\)~be a category with products and pullbacks. Then \(\C\)~has equalizers.
\end{exer}
\begin{proof}
Given \(f,g:A\to B\), construct this pullback:
\begin{diagram}
E		&\rTo^h				&B\\
\dTo<e	&					&\dTo>{\pair{1_B}{1_B}}\\
A		&\rTo_{\pair{f}{g}}	&B\times B
\end{diagram}
We claim that \(e:E\to A\) is an equalizer of \(f\)~and~\(g\). Indeed, since the diagram commutes,
\[\pair{fe}{ge}=\pair{f}{g}e=\pair{1_B}{1_B}h=\pair{h}{h}\]
Therefore \(fe=ge\). And if \(z:Z\to A\) with \(fz=gz\), then this square commutes:
\begin{diagram}
Z		&\rTo^{fz=gz}		&B\\
\dTo<z	&					&\dTo>{\pair{1_B}{1_B}}\\
A		&\rTo_{\pair{f}{g}}	&B\times B
\end{diagram}
Since \(E\)~is a pullback, there exists a unique \(u:Z\to E\) with \(z=eu\).
\end{proof}

\begin{exer}[7]
Let \(\C\)~be a locally small category with all small limits and \(C\in\C\). Then the representable functor
\[\Hom_{\C}(C,-):\C\to\Sets\]
is continuous.
\end{exer}
\begin{proof}
We prove this directly from the definition of limit, not using products and equalizers.\footnote{Compare the proof of Proposition~5.25.}

Write \(H=\Hom_{\C}(C,-)\). Let \(D:\J\to\C\) be a diagram of type~\(\J\) in~\(\C\), and let \(p_j:\limit_j D_j\to D_j\) be a limit for~\(D\) in~\(\C\). We claim \(H(p_j):H(\limit_j D_j)\to H(D_j)\) is a limit for~\(HD\) in~\(\Sets\). Indeed, clearly it is a cone to~\(HD\) in~\(\Sets\) since \(H\)~is a functor. Suppose \((X,x_j)\)~is a cone to~\(HD\) in~\(\Sets\), so the arrows \(x_j:X\to H(D_j)\) form commutative triangles of the following form for \(\alpha:i\to j\) in~\(\J\):
\begin{diagram}[nohug]
		&			&X						&			&\\
		&\ldTo<{x_i}&						&\rdTo>{x_j}&\\
H(D_i)	&			&\rTo_{H(D_{\alpha})}	&			&H(D_j)
\end{diagram}
We must show there is a unique \(u:X\to H(\limit_j D_j)\) in~\(\Sets\) with \(x_j=H(p_j)u\). To this end, observe that for \(x\in X\), \((C,x_j(x))\)~is a cone to~\(D\) in~\(\C\). Indeed, from the above diagram it follows that for \(j\in\J\), \(x_j(x):C\to D_j\) in~\(\C\) and for \(\alpha:i\to j\in\J\), this diagram commutes:
\begin{diagram}[nohug]
		&				&C					&				&\\
		&\ldTo<{x_i(x)}	&					&\rdTo>{x_j(x)}	&\\
D_i		&				&\rTo_{D_{\alpha}}	&				&D_j
\end{diagram}
Let \(u(x)\)~be the unique \(C\to\lim_j D_j\) in~\(\C\) with \(x_j(x)=p_ju(x)\). Then for \(x\in X\),
\[(H(p_j)u)(x)=H(p_j)(u(x))=p_ju(x)=x_j(x)\]
So \(H(p_j)u=x_j\). Moreover, if \(u':X\to H(\limit_j D_j)\) in~\(\Sets\) satisfies \(x_j=H(p_j)u'\), then for \(x\in X\), \(u'(x):C\to\limit_j D_j\) in~\(\C\) with \(x_j(x)=H(p_j)u'(x)\), so \(u'(x)=u(x)\) by uniqueness of~\(u(x)\). Therefore \(u'=u\), so \(u\)~is unique as needed.
\end{proof}

\begin{exer}[9]
Let \(\C\)~be a category with limits of type~\(\J\). There exists a category \(\Diagrams(\J,\C)\) of diagrams of type~\(\J\) in~\(\C\), and a limit functor
\[\limit_{\J}:\Diagrams(\J,\C)\to\C\]
In particular, there exists a product functor
\[\prod_{i\in I}:\Sets^{I}\to\Sets\]
for \(I\)-indexed families of sets.
\end{exer}
\begin{proof}
The objects in \(\Diagrams(\J,\C)\) are functors \(F:\J\to\C\), and the arrows are natural transformations between those functors.\footnote{In other words, \(\Diagrams(\J,\C)\) is just the functor category~\(\C^{\J}\).} More specifically, an arrow \(\theta:F\to G\) between \(F:\J\to\C\) and \(G:\J\to\C\) is a family of arrows \(\theta_j:Fj\to Gj\) for each \(j\in\J\) such that for \(\alpha:i\to j\in\J\), this square commutes:
\begin{diagram}
Fi				&\rTo^{\theta_i}	&Gi\\
\dTo<{F\alpha}	&					&\dTo>{G\alpha}\\
Fj				&\rTo_{\theta_j}	&Gj
\end{diagram}
The identity~\(1_F\) consists of the identity arrows~\(1_{Fj}\) for \(j\in\J\). If \(\theta:F\to G\) and \(\lambda:G\to H\), then \(\lambda\theta:F\to H\) consists of the composite arrows~\(\lambda_j\theta_j\) for \(j\in\J\). Indeed, the diagrams involved obviously commute, and associativity and unity of composition are inherited from~\(\C\).

For \(F:\J\to\C\), let \(\limit F\)~be the vertex of the limit of~\(F\) in~\(\C\). For \(\theta:F\to G\), let \(f_j:\limit F\to Fj\) and \(g_j:\limit G\to Gj\) in~\(\C\) and observe that \(\theta_jf_j:\limit F\to Gj\) is a cone to~\(G\) in~\(\C\). Let \(\limit \theta\)~be the unique \(u_{\theta}:\limit F\to\limit G\) in~\(\C\) such that \(\theta_jf_j=g_ju_{\theta}\). We claim \(\limit\)~is a functor. Indeed, it clearly maps objects to objects and arrows to arrows and preserves domains and codomains of arrows. It is also clear that \(\limit 1_F=1_{\limit F}\). If \(\theta:F\to G\) and \(\lambda:G\to H\), then \(\limit\lambda\theta\) is the unique \(u_{\lambda\theta}:\limit F\to\limit H\) such that \(\lambda_j\theta_j f_j=h_ju_{\lambda\theta}\). Now \(u_{\lambda}u_{\theta}:\limit F\to\limit H\) and it follows from \(\theta_jf_j=g_ju_{\theta}\) and \(\lambda_jg_j=h_ju_{\lambda}\) that \(\lambda_j\theta_jf_j=h_ju_{\lambda}u_{\theta}\). Therefore \(u_{\lambda\theta}=u_{\lambda}u_{\theta}\) by uniqueness of~\(u_{\lambda\theta}\), that is, \(\limit\lambda\theta=(\limit\lambda)(\limit\theta)\), as desired.

Now viewing the set~\(I\) as a discrete category, an \(I\)-indexed family of sets is just a diagram of type~\(I\) in~\(\Sets\), and in this case the limit is just the product. Hence \(\prod\)~is a functor by the above.
\end{proof}

\begin{exer}[12]
In~\(\Pos\), let \([n]=\{0\le\cdots\le n\}\), let \([n]\to[n+1]\) be inclusion, and let the sequence \(S:\omega\to\Pos\) be given by
\[[0]\to[1]\to\cdots[n]\to[n+1]\to\cdots\]
Then \(\limit S=[0]\) and \(\colimit S=\omega\).
\end{exer}
\begin{proof}
Immediate from definitions.
\end{proof}

\section*{Chapter~6}
\begin{rmk}
Let \(\C\)~be a cartesian closed category and \(\eta=\curry{1_{A\times B}}:A\to(A\times B)^B\). For \(f:A\times B\to C\), to see that the diagram
\begin{diagram}[nohug]
(A\times B)^B	&\rTo^{f^B}			&C^B\\
\uTo<{\eta}		&\ruTo>{\curry{f}}	&\\
A				&					&
\end{diagram}
commutes, observe that the diagram
\begin{diagram}[nohug]
C^B\times B				&\rTo^{\eval}			&C\\
\uTo<{f^B\times 1_B}	&						&\uTo>f\\
(A\times B)^B\times B	&\rTo_{\eval}			&A\times B\\
\uTo<{\eta\times 1_B}	&\ruTo>{1_{A\times B}}	&\\
A\times B				&						&
\end{diagram}
commutes and
\[(f^B\times 1_B)\after(\eta\times 1_B)=(f^B\after\eta)\times 1_B\]
This result shows that the transpose of~\(f\) can be obtained by applying~\(f\) after the transposed pairing operation~\(\eta\). The transpose~\(\eta\) is ``universal'' because every transpose can be obtained from it in this way.
\end{rmk}

\begin{rmk}
If \(\C\)~and~\(\D\) are cartesian closed categories, then so is~\(\C\times\D\). Moreover, the terminal object, products, and exponentials (including the evaluation and transpose maps for exponentials) are all computed pointwise.
\end{rmk}
\begin{proof}
Let \(1_{\C}\)~be terminal in~\(\C\) and \(1_{\D}\)~be terminal in~\(\D\). Then for \((C,D)\in\C\times\D\), \((\term_C,\term_D):(C,D)\to(1_{\C},1_{\D})\) is unique, so \(1_{\C\times\D}=(1_{\C},1_{\D})\)~is terminal in~\(\C\times\D\).

Suppose \((A,B),(C,D)\in\C\times\D\). Then \((A\times C,B\times D)\in\C\times\D\) and
\[(A,B)\lTo^{(p_1,p_1)}(A\times C,B\times D)\rTo^{(p_2,p_2)}(C,D)\]
For \(f:(X,Y)\to(A,B)\) and \(g:(X,Y)\to(C,D)\), define
\[\pair{f}{g}=\bigl(\pair{f_1}{g_1},\pair{f_2}{g_2}\bigr):(X,Y)\to(A\times C,B\times D)\]
Then
\[(p_1,p_1)\after\pair{f}{g}=(f_1,f_2)=f\quad\text{and}\quad(p_2,p_2)\after\pair{f}{g}=(g_1,g_2)=g\]
Conversely, for any \(h:(X,Y)\to(A\times C,B\times D)\),
\begin{align*}
\pair{(p_1,p_1)\after h}{(p_2,p_2)\after h}&=\pair{(p_1\after h_1,p_1\after h_2)}{(p_2\after h_1,p_2\after h_2)}\\
	&=\bigl(\pair{p_1\after h_1}{p_2\after h_1},\pair{p_1\after h_2}{p_2\after h_2}\bigr)\\
	&=(h_1,h_2)\\
	&=h
\end{align*}
Therefore \((A,B)\times(C,D)=(A\times C,B\times D)\) is a product in~\(\C\times\D\).

We claim \((C^A,D^B)\)~is an exponential for \((A,B)\)~and~\((C,D)\) in~\(\C\times\D\). Indeed, let \(\eval_{C^A}:C^A\times A\to C\) and \(\eval_{D^B}:D^B\times B\to D\) be evaluation arrows in \(\C\)~and~\(\D\) respectively and define
\[\eval=(\eval_{C^A},\eval_{D^B}):(C^A,D^B)\times(A,B)\to(C,D)\]
For \(f:(X,Y)\times(A,B)\to(C,D)\), \(f_1:X\times A\to C\) and \(f_2:Y\times B\to D\), so \(\curry{f_1}:X\to C^A\) and \(\curry{f_2}:Y\to D^B\). Define
\[\curry{f}=(\curry{f_1},\curry{f_2}):(X,Y)\to(C^A,D^B)\]
Then
\begin{align*}
\eval\after(\curry{f}\times 1_{(A,B)})&=(\eval_{C^A},\eval_{D^B})\after((\curry{f_1},\curry{f_2})\times(1_A,1_B))\\
	&=(\eval_{C^A},\eval_{D^B})\after(\curry{f_1}\times1_A,\curry{f_2}\times1_B)\\
	&=(f_1,f_2)\\
	&=f
\end{align*}
Conversely, for \(h:(X,Y)\to(C^A,D^B)\),
\[\uncurry{h}=\eval\after(h\times1_{(A,B)})=(\eval_{C^A},\eval_{D^B})\after(h_1\times1_A,h_2\times1_B)=(\uncurry{h_1},\uncurry{h_2})\]
So
\[\curry{\uncurry{h}}=\curry{(\uncurry{h_1},\uncurry{h_2})}=(\curry{\uncurry{h_1}},\curry{\uncurry{h_2}})=(h_1,h_2)=h\]
Therefore \((C,D)^{(A,B)}=(C^A,D^B)\) is an exponential in~\(\C\times\D\) as claimed.
\end{proof}

\begin{exer}[2]
Let \(\C\)~be a cartesian closed category and \(A,B,C\in\C\).
\begin{enumerate}[itemsep=0pt]
\item[(a)] \((A\times B)^C\iso A^C\times B^C\)
\item[(b)] \((A^B)^C\iso A^{B\times C}\)
\end{enumerate}
\end{exer}
\begin{proof}\
\begin{enumerate}[itemsep=0pt]
\item[(a)] We prove that \(A^C\times B^C\)~is an exponential for \(A\times B\)~and~\(C\), from which the isomorphism follows by uniqueness of exponentials under the universal mapping property.

Let \(\eval_A:A^C\times C\to A\) and \(\eval_B:B^C\times C\to B\) be evaluation arrows, and let
\begin{align*}
\alpha:(A^C\times B^C)\times(C\times C)&\iso(A^C\times C)\times(B^C\times C)\\
	\pair{\pair{w}{x}}{\pair{y}{z}}&\mapsto\pair{\pair{w}{y}}{\pair{x}{z}}
\end{align*}
for generalized elements \(w,x,y,z\). Observe
\[\alpha:\pair{w}{x}\times\pair{y}{z}\mapsto\pair{w\times y}{x\times z}\]
Define \(\eval:(A^C\times B^C)\times C\to A\times B\) by
\[\eval=(\eval_A\times\eval_B)\after\alpha\after(1_{A^C\times B^C}\times\pair{1_C}{1_C})\]
We claim \((A^C\times B^C,\eval)\)~is an exponential for \(A\times B\)~and~\(C\); that is, for all \(f:Z\times C\to A\times B\), there is a unique \(\curry{f}:Z\to A^C\times B^C\) such that \(\eval\after(\curry{f}\times1_C)=f\):
\begin{diagram}[nohug]
A^C\times B^C		&&(A^C\times B^C)\times C		&\rTo^{\eval}	&A\times B\\
\uTo<{\curry{f}}	&&\uTo<{\curry{f}\times1_C}		&\ruTo>f		&\\
Z					&&Z\times C						&				&
\end{diagram}
Indeed, suppose \(f:Z\times C\to A\times B\). Let \(f_1=p_1\after f\) and \(f_2=p_2\after f\):
\begin{diagram}[nohug]
	&			&Z\times C	&				&\\
	&\ldTo<{f_1}&\dTo>f		&\rdTo>{f_2}	&\\
A	&\lTo^{p_1}	&A\times B	&\rTo^{p_2}		&B
\end{diagram}
Then there exist unique \(\curry{f_1}:Z\to A^C\) and \(\curry{f_2}:Z\to B^C\) with \(\eval_A\after(\curry{f_1}\times1_C)=f_1\) and \(\eval_B\after(\curry{f_2}\times1_C)=f_2\). Define \(\curry{f}=\pair{\curry{f_1}}{\curry{f_2}}:Z\to A^C\times B^C\). Then
\begin{align*}
\eval\after(\curry{f}\times1_C)&=(\eval_A\times\eval_B)\after\alpha\after(1_{A^C\times B^C}\times\pair{1_C}{1_C})\after(\pair{\curry{f_1}}{\curry{f_2}}\times1_C)\\
	&=(\eval_A\times\eval_B)\after\alpha\after(\pair{\curry{f_1}}{\curry{f_2}}\times\pair{1_C}{1_C})\\
	&=(\eval_A\times\eval_B)\after\pair{\curry{f_1}\times1_C}{\curry{f_2}\times1_C}\\
	&=\pair{\eval_A\after(\curry{f_1}\times1_C)}{\eval_B\after(\curry{f_2}\times1_C)}\\
	&=\pair{f_1}{f_2}\\
	&=f
\end{align*}
Finally, \(\curry{f}\)~is unique in satisfying this property since \(\curry{f_1}\)~and~\(\curry{f_2}\) are unique. This establishes the claim.
\item[(b)] We exhibit isomorphisms between \((A^B)^C\)~and~\(A^{B\times C}\) directly. Define \(g:(A^B)^C\to A^{B\times C}\) and \(h:A^{B\times C}\to(A^B)^C\) by
\[g=\curry{(\uncurry{\eval}\after\alpha)}\qquad h=\curry{\curry{(\eval\after\inv{\alpha})}}\]
where \(\alpha:Z\times(B\times C)\iso(Z\times C)\times B\).\footnote{We do not distinguish notationally between the different evaluation, transpose, inverse transpose, and isomorphism arrows involved. However, the context makes clear which ones are intended.} We claim that \(g\)~and~\(h\) are mutually inverse, from which it follows that they are isomorphisms.

By the universal mapping property for exponentials applied twice,
\[h\after g=1_{(A^B)^C}\iff\eval\after((\eval\after((h\after g)\times 1_C))\times 1_B)=\eval\after(\eval\times 1_B)\]
Observe
\begin{align*}
\eval\after((\eval\after((h\after g)\times 1_C))\times 1_B)&=\eval\after((\eval\after(h\times1_C)\after(g\times1_C))\times 1_B)\\
	&=\eval\after((\uncurry{h}\after(g\times1_C))\times 1_B)\\
	&=\eval\after(\uncurry{h}\times1_B)\after((g\times1_C)\times1_B)\\
	&=\uncurry{\uncurry{h}}\after((g\times1_C)\times1_B)\\
	&=\eval\after\inv{\alpha}\after((g\times1_C)\times1_B)\\
	&=\eval\after\inv{\alpha}\after\alpha(g\times(1_B\times1_C))\after\inv{\alpha}\\
	&=\eval\after(g\times1_{B\times C})\after\inv{\alpha}\\
	&=\uncurry{g}\after\inv{\alpha}\\
	&=\uncurry{\eval}\after\alpha\after\inv{\alpha}\\
	&=\uncurry{\eval}\\
	&=\eval\after(\eval\times1_B)
\end{align*}
Therefore \(h\after g=1_{(A^B)^C}\). Similarly \(g\after h=1_{A^{B\times C}}\). So \(g\)~and~\(h\) are mutually inverse as claimed.\qedhere
\end{enumerate}
\end{proof}
\begin{rmk}
In~\(\Sets\), this exercise (circuitously) justifies the familiar exponent laws \((ab)^c=a^cb^c\) and \((a^b)^c=a^{bc}\) for \(a,b,c\in\N\).
\end{rmk}

\begin{exer}[4]
\(\Mon\)~is not cartesian closed.
\end{exer}
\begin{proof}
Suppose \(\Mon\)~is cartesian closed. Let \(M\)~and~\(N\) be any monoids with distinct homomorphisms \(f,g:M\to N\) (for example, take \(M=N=(\N,+)\), \(f=0\), and \(g=1\)). Define
\begin{align*}
f':1\times M&\to N&g':1\times M&\to N\\
(0,m)&\mapsto f(m)&(0,m)&\mapsto g(m)
\end{align*}
Then clearly \(f'\)~and~\(g'\) are also distinct homomorphisms. By assumption there is an exponential~\(N^M\) and transpose homomorphisms
\[\curry{f'}:1\to N^M\qquad\text{and}\qquad\curry{g'}:1\to N^M\]
However since \(N^M\)~is a monoid, there is only one homomorphism \(1\to N^M\) (the identity element must be mapped to the identity element), so we must have \(\curry{f'}=\curry{g'}\). Therefore
\[f'=\uncurry{\curry{f'}}=\uncurry{\curry{g'}}=g'\]
---contradicting that \(f'\ne g'\).\footnote{See also Exercise~9.}
\end{proof}

\begin{exer}[9]
Let \(\C\)~be a cartesian closed category and \(A,B\in\C\). Then there is a bijective correspondence between arrows \(A\to B\) and arrows \(1\to B^A\).
\end{exer}
\begin{proof}
This follows from
\[\Hom_{\C}(A,B)\iso\Hom_{\C}(1\times A,B)\iso\Hom_{\C}(1,B^A)\]
The second isomorphism follows from the definition of the exponential. For the first isomorphism, let
\[1\lTo^{\term_{1\times A}}1\times A\rTo^p A\]
be projections, and let \(q=\pair{\term_A}{1_A}:A\to1\times A\). Then \(p\after q=1_A\), and conversely
\[q\after p=\pair{\term_A}{1_A}\after p=\pair{\term_A\after p}{p}=\pair{\term_{1\times A}}{p}=1_{1\times A}\]
so \(A\iso 1\times A\), and the functor \(\Hom_{\C}(-,B)\) preserves this isomorphism.
\end{proof}

\begin{exer}[12]
Let \(\C\)~be a cartesian closed category and \(C\in\C\). Exponentiation with base object~\(C\) gives a contravariant functor \(C^{(-)}:\Cop\to\C\).
\end{exer}
\begin{proof}
For objects \(A\in\C\), define \(C^{(-)}(A)=C^A\). For arrows \(f:A\to B\) in~\(\C\), define
\[C^{(-)}(f)=C^f=\curry{(\eval_{C^B}\after(1_{C^B}\times f))}:C^B\to C^A\]
where \(\eval_{C^B}:C^B\times B\to C\) is evaluation, so the following diagram commutes:
\begin{diagram}
C^B\times B				&\rTo^{\eval_{C^B}}		&C\\
\uTo<{1_{C^B}\times f}	&						&\uTo>{\eval_{C^A}}\\
C^B\times A				&\rTo_{C^f\times 1_A}	&C^A\times A
\end{diagram}
Then \(C^{(-)}\)~maps objects to objects and arrows to arrows and preserves domains and codomains of arrows in~\(\Cop\). Also
\[C^{1_A}=\curry{(\eval_{C^A}\after(1_{C^A}\times1_A))}=\curry{(\eval_{C^A}\after1_{C^A\times A})}=\curry{\eval_{C^A}}=1_{C^A}\]
so \(C^{(-)}\)~preserves identities. If \(f:X\to Y\) and \(g:Y\to Z\), then the following diagram commutes (the upper left and lower right inner squares are just the squares above for \(g\)~and~\(f\), respectively, and the other two inner squares are trivial):
\begin{diagram}
C^Z\times Z				&\rTo^{\eval_{C^Z}}	&C						&\rTo^{1_C}			&C\\
\uTo<{1_{C^Z}\times g}	&					&\uTo>{\eval_{C^Y}}		&					&\uTo>{1_C}\\
C^Z\times Y				&\rTo_{C^g\times1_Y}&C^Y\times Y			&\rTo_{\eval_{C^Y}}	&C\\
\uTo<{1_{C^Z}\times f}	&					&\uTo>{1_{C^Y}\times f}	&					&\uTo>{\eval_{C^X}}\\
C^Z\times X				&\rTo_{C^g\times1_X}&C^Y\times X			&\rTo_{C^f\times1_X}&C^X\times X
\end{diagram}
On the left we have
\[(1_{C^Z}\times g)\after(1_{C^Z}\times f)=1_{C^Z}\times(g\after f)\]
On the bottom we have
\[(C^f\times1_X)\after(C^g\times1_X)=(C^f\after C^g)\times1_X\]
Commutativity of the diagram therefore implies
\[\uncurry{C^{g\after f}}=\eval_{C^Z}\after(1_{C^Z}\times(g\after f))=\eval_{C^X}\after((C^f\after C^g)\times1_X)=\uncurry{C^f\after C^g}\]
It follows that \(C^{g\after f}=C^f\after C^g\) and hence \(C^{(-)}\)~preserves composites in~\(\Cop\). This completes the proof that \(C^{(-)}\)~is a contravariant functor.
\end{proof}

\begin{exer}[13]
Let \(\C\)~be a cartesian closed category with coproducts and \(A,B,C\in\C\). Then
\[(A+B)\times C\iso(A\times C)+(B\times C)\]
\end{exer}
\begin{proof}
We prove that \((A+B)\times C\)~is a coproduct of \(A\times C\)~and~\(B\times C\), from which the isomorphism follows by uniqueness of coproducts under the universal mapping property.

Observe the injections
\[A\times C\rTo^{i_1\times1_C}(A+B)\times C\lTo^{i_2\times1_C}B\times C\]
If \(f:A\times C\to Z\) and \(g:B\times C\to Z\), then \(\curry{f}:A\to Z^C\) and \(\curry{g}:B\to Z^C\), so \(\copair{\curry{f}}{\curry{g}}:A+B\to Z^C\). Define
\[p=\uncurry{\copair{\curry{f}}{\curry{g}}}:(A+B)\times C\to Z\]
Then
\begin{align*}
p\after(i_1\times1_C)&=\uncurry{\copair{\curry{f}}{\curry{g}}}\after(i_1\times1_C)\\
	&=\eval\after(\copair{\curry{f}}{\curry{g}}\times1_C)\after(i_1\times1_C)\\
	&=\eval\after((\copair{\curry{f}}{\curry{g}}\after i_1)\times1_C)\\
	&=\eval\after(\curry{f}\times1_C)\\
	&=f
\end{align*}
Similarly \(p\after(i_2\times1_C)=g\). Moreover, if \(q:(A+B)\times C\to Z\) is arbitrary, then \(\curry{q}:A+B\to Z^C\), so
\begin{align*}
\curry{q}&=\copair{\curry{q}\after i_1}{\curry{q}\after i_2}\\
	&=\copair{\curry{(q\after(i_1\times1_C))}}{\curry{(q\after(i_2\times1_C))}}
\end{align*}
because
\[\eval\after((\curry{q}\after i_k)\times1_C)=\eval\after(\curry{q}\times1_C)\after(i_k\times1_C)=q\after(i_k\times1_C)\]
It follows that
\[q=\uncurry{\copair{\curry{(q\after(i_1\times1_C))}}{\curry{(q\after(i_2\times1_C))}}}\]
from which it is immediate that \(p\)~is unique in satisfying the equations above. Therefore \((A+B)\times C\)~is indeed a coproduct as desired. 
\end{proof}
\begin{rmk}
In~\(\Sets\), this exercise (circuitously) justifies the familiar distributive law \((a+b)c=ac+bc\) for \(a,b,c\in\N\).
\end{rmk}

\section*{Chapter~7}
\begin{rmk}
A natural transformation \(\theta:F\to G\) between two functors \(F:\C\to\D\) and \(G:\C\to\D\) looks like this:
\begin{center}
\begin{tikzcd}[sep=huge] % not doing tab alignment on this one
A \ar[r, "f"] \ar[rd, "g\after f"' name=gf] & B \ar[d, "g" name=g] & & \\
 & C & & \\
 & & GA \ar[r, "Gf" name=Gf] \ar[rd, "G(g\after f)"'] & GB \ar[d, "Gg"] \\
 & & & GC \\
FA \ar[r, "Ff" name=Ff] \ar[rd, "F(g\after f)"'] \ar[rruu, dashed, bend left, "\theta_A"] & FB \ar[d, "Fg"] \ar[rruu, dashed, bend left, end anchor=south west, crossing over, "\theta_B"] & & \\
 & FC \ar[rruu, dashed, bend left, start anchor=north east, "\theta_C"] & &
 \ar[from=gf, to=Ff, Rightarrow, shorten <=2em, shorten >=5em, bend right=20, "F"{', inner sep=1.25em, yshift=3em}]
 \ar[from=g, to=Gf, Rightarrow, shorten=3em, bend left=10, "G" inner sep=0.75em]
\end{tikzcd}
\end{center}
\end{rmk}

\begin{rmk}
For functors
\begin{diagram}
\B&\rTo^R&\C&\pile{\rTo^F\\\rTo_G}&\D&\rTo^S&\E
\end{diagram}
if \(F\iso G\), then \(SFR\iso SGR\).
\end{rmk}
\begin{proof}
If \(\varphi:F\to G\) is a natural transformation, define \(\psi:SFR\to SGR\) by
\[\psi_B=S(\varphi_{RB}):SFRB\to SGRB\]
If \(f:B\to B'\in\B\), then applying~\(S\) to the commutative diagram
\begin{diagram}
FRB			&\rTo^{\varphi_{RB}}	&GRB\\
\dTo<{FRf}	&						&\dTo>{GRf}\\
FRB'		&\rTo_{\varphi_{RB'}}	&GRB'
\end{diagram}
yields the commutative diagram
\begin{diagram}
SFRB		&\rTo^{\psi_B}		&SGRB\\
\dTo<{SFRf}	&					&\dTo>{SGRf}\\
SFRB'		&\rTo_{\psi_{B'}}	&SGRB'
\end{diagram}
so \(\psi\)~is a natural transformation. If \(\varphi\)~is a natural isomorphism, then so is~\(\psi\) by Lemma~7.11.
\end{proof}

\begin{rmk}
If \(F_1\iso G_1\) and \(F_2\iso G_2\), then by the previous remark \(F_1F_2\iso G_1G_2\).
\end{rmk}

\begin{rmk}
The previous remarks are useful in establishing natural isomorphisms. For example, in a locally small category with products, it follows from
\[A\times(B\times C)\iso(A\times B)\times C\]
being natural in \(A,B,C\) that
\[\Hom(A\times(B\times C),X)\iso\Hom((A\times B)\times C,X)\]
is also natural in \(A,B,C\).
\end{rmk}

\begin{rmk}
The bifunctor lemma (Lemma~7.14) just says that a map \(F:\A\times\B\to\C\) is a functor if and only if it is functorial in each argument and the functors in each argument induce natural transformations between the functors in the other argument (this is the ``interchange law'').

More specifically, \(F\)~is a functor if and only if for each fixed \(A\in\A\) and \(B\in\B\), \(F(A,-)\) and~\(F(-,B)\) are functors and for any \(\alpha:A\to A'\) and \(\beta:B\to B'\), \(F(-,\beta)\)~is a natural transformation from~\(F(-,B)\) to~\(F(-,B')\) and \(F(\alpha,-)\)~is a natural transformation from~\(F(A,-)\) to~\(F(A',-)\):
\begin{diagram}
F(A,B)				&\rTo^{F(A,\beta)}	&F(A,B')\\
\dTo<{F(\alpha,B)}	&					&\dTo>{F(\alpha,B')}\\
F(A',B)				&\rTo_{F(A',\beta)}	&F(A',B')
\end{diagram}
This just means that the two paths these functors induce between the objects \(F(A,B)\) and~\(F(A',B')\) agree.
\end{rmk}

\begin{rmk}
The fact that \(\Par\eqv\Setsp\) can be used to explain the use of sentinel values in computer programming. The points introduced by the equivalence functor \(F:\Par\to\Setsp\) are just sentinel values used to signal where partial functions are undefined.
\end{rmk}

\begin{rmk}
If \(B\)~is a finite Boolean algebra and \(b\in B\) with \(b\ne 0\), then there is an atom \(a\in A(B)\) with \(a\le b\).
\end{rmk}
\begin{proof}
By induction on the number of elements less than~\(b\).
\end{proof}

\begin{rmk}[Lemma~7.33]
If \(B\)~is a finite Boolean algebra, then
\begin{enumerate}
\item[(i)] \(b=\bigjoin\{\,a\in A(B)\mid a\le b\,\}\)
\item[(ii)] If \(a\in A(B)\) and \(a\le b\join b'\), then \(a\le b\) or \(a\le b'\).
\end{enumerate}
\end{rmk}
\begin{proof}
For~(i), let \(c=\bigjoin\{\,a\in A(B)\mid a\le b\,\}\). Clearly \(c\le b\). If \(b\not\le c\), then \(b\meet\compl c\ne 0\), so by the previous remark there is \(a\in A(B)\) with \(a\le b\meet\compl c\). Now \(a\le b\) so \(a\le c\), and also \(a\le\compl c\), so \(a\le c\meet\compl c=0\), contradicting that \(a\ne 0\).

For~(ii), if \(a\not\le b\) and \(a\not\le b'\), then \(a\le\compl b\) and \(a\le\compl b'\) (Lemma~7.32), so
\[a\le\compl b\meet\compl b'=\compl(b\join b')\]
and hence \(a\not\le b\join b'\).
\end{proof}

\begin{exer}[1]
Let \(\mathcal{F}=\powBA\after\dual{\ult}:\BA\to\Setsop\to\BA\). For a Boolean algebra~\(B\), define \(\phi_B:B\to\mathcal{F}(B)\) by
\[\phi_B(b)=\{\,V\in\ult(B)\mid b\in V\,\}\]
Then \(\phi_B\)~is a Boolean homomorphism, and for any Boolean homomorphism \(h:A\to B\), the following diagram commutes:
\begin{diagram}
A		&\rTo^{\phi_A}	&\mathcal{F}(A)\\
\dTo<h	&				&\dTo>{\mathcal{F}(h)}\\
B		&\rTo_{\phi_B	}&\mathcal{F}(B)
\end{diagram}
\end{exer}
\begin{proof}
It is immediate from ultrafilter properties that \(\phi_B\)~is a homomorphism. For \(a\in A\), we have
\begin{align*}
(\phi_B\after h)(a)&=\phi_B(h(a))\\
	&=\{\,V\in\ult(B)\mid h(a)\in V\,\}\\
	&=\{\,V\in\ult(B)\mid a\in\inv{h}(V)\in\ult(A)\,\}\\
	&=\inv{\dual{\ult}(h)}\{\,U\in\ult(A)\mid a\in U\,\}\\
	&=(\powBA(\dual{\ult}(h)))(\phi_A(a))\\
	&=\mathcal{F}(h)(\phi_A(a))\\
	&=(\mathcal{F}(h)\after\phi_A)(a)\qedhere
\end{align*}
\end{proof}

\begin{exer}[2]
The homomorphism~\(\phi_B\) from the previous exercise is injective.
\end{exer}
\begin{proof}
If \(a,b\in B\) and \(a\ne b\), then we may assume \(a\not\le b\), so \(a\meet\compl b\ne 0\). Now \(\up(a\meet\compl b)\)~is a proper filter, which is contained in an ultrafilter~\(V\) by the ultrafilter theorem. It follows that \(a\in V\) but \(b\not\in V\), so \(\phi_B(a)\ne\phi_B(b)\).
\end{proof}

\begin{exer}[3]
The homomorphism~\(\phi_B\) from the previous exercise is bijective if \(B\)~is finite.
\end{exer}
\begin{proof}
By Lemmas 7.32 and~7.33, the mapping \(\psi_B:\mathcal{F}(B)\to B\) defined by
\[\psi_B(S)=\bigjoin\{\,\bigmeet_{b\in V}b\mid V\in S\,\}\]
is left inverse to~\(\phi_B\).
\end{proof}

\begin{exer}[4]
The forgetful functors
\[\Grp\rTo^{U}\Mon\rTo^{V}\Sets\]
have the following properties:
\begin{center}
\begin{tabular}{|r|c|c|}
\hline
						&\(U\)	&\(V\)\\
\hline
Injective on objects	&Yes	&No\\
Injective on arrows		&Yes	&Yes\\
Surjective on objects	&No		&No\\
Surjective on arrows	&No		&No\\
Faithful				&Yes	&Yes\\
Full					&Yes	&No\\
\hline
\end{tabular}
\end{center}
\end{exer}

\begin{exer}[7]
A natural transformation is an isomorphism if and only if each of its components is an isomorphism.
\end{exer}
\begin{proof}
Let \(F,G:\C\to\D\) be functors and \(\vartheta:F\to G\) a natural transformation with components \(\vartheta_C:FC\to GC\) for all \(C\in\C\). We claim \(\vartheta\)~is an iso in~\(\D^{\C}\) if and only if \(\vartheta_C\)~is an iso in~\(\D\) for all \(C\in\C\).

Suppose \(\vartheta\)~is an iso with inverse \(\psi:G\to F\), so \(\psi\after\vartheta=1_F\) and \(\vartheta\after\psi=1_G\). Since composites and identities in~\(\D^{\C}\) are defined componentwise, it is immediate that \(\psi_C\)~is an inverse of~\(\vartheta_C\), so \(\vartheta_C\)~is an iso, for all \(C\in\C\).

Suppose conversely that for all \(C\in\C\), \(\vartheta_C:FC\to GC\)~is an iso, so there is an inverse \(\psi_C:GC\to FC\) with \(\psi_C\after\vartheta_C=1_{FC}\) and \(\vartheta_C\after\psi_C=1_{GC}\). We claim the family~\(\psi\) is a natural transformation from~\(G\) to~\(F\). Indeed, for \(\alpha:B\to C\) in~\(\C\), we know \(\vartheta_C\after F\alpha=G\alpha\after\vartheta_B\):
\begin{diagram}
FB				&\rTo^{\vartheta_B}		&GB\\
\dTo<{F\alpha}	&						&\dTo>{G\alpha}\\
FC				&\rTo_{\vartheta_C}		&GC
\end{diagram}
Applying \(\psi_C\)~on the left and \(\psi_B\)~on the right, we obtain \(\psi_C\after G\alpha=F\alpha\after\psi_B\):
\begin{diagram}
GB				&\rTo^{\psi_B}		&FB\\
\dTo<{G\alpha}	&					&\dTo>{F\alpha}\\
GC				&\rTo_{\psi_C}		&FC
\end{diagram}
So \(\psi\in\D^{\C}\). It is immediate that \(\psi\after\vartheta=1_F\) and \(\vartheta\after\psi=1_G\), so \(\vartheta\)~is an iso.
\end{proof}
\begin{rmk}
If a natural transformation consists of monos, it is a mono, but the converse is not true.
\end{rmk}

\begin{exer}[9]
The function
\begin{align*}
\eta_A:A&\to\pow\pow(A)\\
a&\mapsto\{\,X\subseteq A\mid a\in X\,\}
\end{align*}
is a natural transformation from~\(1_{\Sets}\) to~\(\pow\pow\), where \(\pow\)~is the contravariant powerset functor.
\end{exer}
\begin{proof}
If \(f:A\to B\) is a function, we claim the following diagram commutes:
\begin{diagram}
A		&\rTo^{\eta_A}	&\pow\pow(A)\\
\dTo<f	&				&\dTo>{\pow\pow(f)}\\
B		&\rTo_{\eta_B}	&\pow\pow(B)
\end{diagram}
By definition,
\begin{align*}
\pow(f):\pow(B)&\to\pow(A)\\
Y&\mapsto\inv{f}(Y)=\{\,x\in A\mid f(x)\in Y\,\}
\end{align*}
so
\begin{align*}
\pow\pow(f):\pow\pow(A)&\to\pow\pow(B)\\
\mathcal{C}&\mapsto\inv{(\inv{f})}(\mathcal{C})=\{\,Y\subseteq B\mid\inv{f}(Y)\in\mathcal{C}\,\}
\end{align*}
Therefore for \(x\in A\),
\begin{align*}
(\eta_B\after f)(x)&=\eta_B(f(x))\\
	&=\{\,Y\subseteq B\mid f(x)\in Y\,\}\\
	&=\{\,Y\subseteq B\mid x\in\inv{f}(Y)\,\}\\
	&=\{\,Y\subseteq B\mid \inv{f}(Y)\in\eta_A(x)\,\}\\
	&=\pow\pow(f)(\eta_A(x))\\
	&=(\pow\pow(f)\after\eta_A)(x)\qedhere
\end{align*}
\end{proof}
\begin{rmk}
The function~\(\eta_A\) is actually a natural embedding since if \(x\ne y\), then \(\{x\}\in\eta_A(x)-\eta_A(y)\), so \(\eta_A(x)\ne\eta_A(y)\).
\end{rmk}

\begin{exer}[10]
Let \(\C\)~be a locally small category. There exists a functor
\[\Hom:\Cop\times\C\to\Sets\]
inducing the familiar representable functors
\[\Hom(C,-):\C\to\Sets\qquad\Hom(-,C):\Cop\to\Sets\]
\end{exer}
\begin{proof}
By the bifunctor lemma (Lemma~7.14), it is sufficient to prove that the representable functors satisfy the ``interchange law'', that is, for all \(\alpha:A'\to A\) and \(\beta:B\to B'\), the following diagram commutes:
\begin{diagram}
\Hom(A,B)				&\rTo^{\Hom(\alpha,B)}	&\Hom(A',B)\\
\dTo<{\Hom(A,\beta)}	&						&\dTo>{\Hom(A',\beta)}\\
\Hom(A,B')				&\rTo_{\Hom(\alpha,B')}	&\Hom(A',B')
\end{diagram}
Indeed, for \(f:A\to B\),
\begin{align*}
(\,\Hom(\alpha,B')\after\Hom(A,\beta)\,)(f)&=\Hom(\alpha,B')(\,\Hom(A,\beta)(f)\,)\\
	&=\Hom(\alpha,B')(\beta\after f)\\
	&=\beta\after f\after\alpha\\
	&=\Hom(A',\beta)(f\after\alpha)\\
	&=\Hom(A',\beta)(\,\Hom(\alpha,B)(f)\,)\\
	&=(\,\Hom(A',\beta)\after\Hom(\alpha,B)\,)(f)\qedhere
\end{align*}
\end{proof}

\begin{exer}[12]
If \(\C\eqv\D\) and \(\C\)~has binary products, so does~\(\D\).
\end{exer}
\begin{proof}
By the characterization of equivalence (Proposition~7.26), there exists a functor \(F:\C\to\D\) which is fully faithful and essentially surjective on objects.

If \(X,Y\in\D\), fix \(A,B\in\C\) with \(\vartheta_X:F(A)\iso X\) and \(\vartheta_Y:F(B)\iso Y\). In~\(\C\), there is a product diagram
\[A\lTo^{p_1}A\times B\rTo^{p_2}B\]
Applying~\(F\) to this diagram, we obtain
\[X\lTo^{\vartheta_X}F(A)\lTo^{F(p_1)}F(A\times B)\rTo^{F(p_2)}F(B)\rTo^{\vartheta_Y}Y\]
We claim this is a product diagram of \(X\)~and~\(Y\) in~\(\D\).

Indeed, for \(Z\in\D\) with \(x:Z\to X\) and \(y:Z\to Y\), fix \(C\in\C\) with \(\vartheta_Z:F(C)\iso Z\). Also fix \(a:C\to A\) and \(b:C\to B\) with \(F(a)=\inv{\vartheta_X}\after x\after\vartheta_Z\) and \(F(b)=\inv{\vartheta_Y}\after y\after\vartheta_Z\). In~\(\C\), there is a unique pair \(p=\pair{a}{b}:C\to A\times B\) with \(p_1\after p=a\) and \(p_2\after p=b\):
\begin{diagram}[nohug]
	&			&C			&			&\\
	&\ldTo<a	&\dTo<p		&\rdTo>b	&\\
A	&\lTo_{p_1}	&A\times B	&\rTo_{p_2}	&B
\end{diagram}
Applying \(F\)~to this diagram, we obtain
\begin{diagram}[nohug,size=3.5em,tight]
	&					&Z		&					&				&					&Z		&					&\\
	&\ldTo<x			&		&\luTo>{\vartheta_Z}&				&\ruTo^{\vartheta_Z}&		&\rdTo>y			&\\
X	&					&		&					&F(C)			&					&		&					&Y\\
	&\luTo<{\vartheta_X}&		&\ldTo<{F(a)}		&\dTo<{F(p)}	&\rdTo>{F(b)}		&		&\ruTo>{\vartheta_Y}&\\
	&					&F(A)	&\lTo_{F(p_1)}		&F(A\times B)	&\rTo_{F(p_2)}		&F(B)	&					&
\end{diagram}
It is immediate from this diagram that \(\pair{x}{y}=F(p)\after\inv{\vartheta_Z}\), which is unique since \(p\)~is unique and \(F\)~is fully faithful.
\end{proof}

\begin{exer}[13]
The ``size'' of a category is respected by isomorphism but not by equivalence. For example, we know \(\Ordf\eqv\Setsf\), but \(\Ordf\)~is countably infinite while \(\Setsf\)~is not even small.
\end{exer}

\begin{exer}[17]
Let \(I\)~be a set. Then
\[\Sets^I\eqv\Sets/I\]
and this equivalence is ``natural'' in the sense that for any function \(f:J\to I\), the following diagram commutes up to natural isomorphism, where \(\Sets^f\) is the reindexing functor, and \(\pull{f}\)~is the pullback functor:
\begin{diagram}
\Sets^I			&\rTo	&\Sets/I\\
\dTo<{\Sets^f}	&		&\dTo>{\pull{f}}\\
\Sets^J			&\rTo	&\Sets/J
\end{diagram}
\end{exer}
\begin{proof}
Define \(\Phi_I:\Sets^I\to\Sets/I\) as follows:
\begin{itemize}
\item Objects: for an indexed family of sets \((A_i)_{i\in I}\), let \(p_i:A_i\to I\) be constant with \(p_i(x)=i\) for all \(x\in A_i\), and define the ``indexing projection''
\[\Phi_I((A_i)_{i\in I})=\pi_I=[p_i]:\coprod_{i\in I}A_i\to I\]
where we take \(\coprod_{i\in I}A_i=\bigunion_{i\in I}(A_i\times\{i\})\).
\item Arrows: for an indexed family of functions \((f_i:A_i\to B_i)_{i\in I}\), define
\[\Phi_I((f_i:A_i\to B_i)_{i\in I})=[\mu_i\after f_i]:\coprod_{i\in I}A_i\to\coprod_{i\in I}B_i\]
where \(\mu_i:B_i\to\coprod_{i\in I}B_i\) is the \(i\)-th coproduct injection.
\end{itemize}
It is immediate that \(\Phi_I\)~is a functor. Define \(\Psi_I:\Sets/I\to\Sets^I\) as follows:
\begin{itemize}
\item Objects: for a function \(\alpha:X\to I\), define
\[\Psi_I(\alpha)=(\inv{\alpha}(i))_{i\in I}\]
\item Arrows: for functions \(\alpha:X\to I\), \(\beta:Y\to I\), and \(\gamma:X\to Y\) with \(\alpha=\beta\after\gamma\), define
\[\Psi_I(\gamma)=\bigl(\,\gamma|_{\inv{\alpha}(i)}:\inv{\alpha}(i)\to\inv{\beta}(i)\,\bigr)_{i\in I}\]
\end{itemize}
It is also immediate that \(\Psi_I\)~is a functor, \(\Psi_I\after\Phi_I=1_{\Sets^I}\), and \(\Phi_I\after\Psi_I\iso 1_{\Sets/I}\), where for \(\alpha:X\to I\) in~\(\Sets/I\), a natural isomorphism from~\(\alpha\) to~\((\Phi_I\after\Psi_I)(\alpha)\) is given by \(x\mapsto(x,\alpha(x))\). Therefore \(\Sets^I\eqv\Sets/I\).

Now suppose \(f:J\to I\) is a function. Define
\begin{align*}
\Sets^f:\Sets^I&\to\Sets^J\\
(A_i)_{i\in I}&\mapsto(A_{f(j)})_{j\in J}\\
(f_i:A_i\to B_i)_{i\in I}&\mapsto(f_{f(j)}:A_{f(j)}\to B_{f(j)})_{j\in J}
\end{align*}
It is immediate that \(\Sets^f\)~is a functor (the ``reindexing functor''). We already know that pullback \(\pull{f}:\Sets/I\to\Sets/J\) is a functor (Proposition~5.10). We claim that \(\Phi_J\after\Sets^f=\pull{f}\after\Phi_I\), which is equivalent up to natural isomorphism to \(\Sets^f=\Psi_J\after\pull{f}\after\Phi_I\). The latter follows from the pullback diagram
\begin{diagram}
\coprod_{j\in J}A_{f(j)}	&\rTo	&\coprod_{i\in I}A_i\\
\dTo<{\pi_J}				&		&\dTo>{\pi_I}\\
J							&\rTo_f	&I
\end{diagram}
where the upper arrow is the reindexing function \((x,j)\mapsto(x,f(j))\).
\end{proof}
\begin{rmk}
We already know that \(\Sets^I\iso\prod_{i\in I}\Sets\) (Example~7.15), so this result shows that \(\Sets/I\eqv\prod_{i\in I}\Sets\). In particular, \(\Sets/2\eqv\Sets\times\Sets\). Since \(\Sets\times\Sets\)~is cartesian closed (by the remark in Chapter~6 above), this implies \(\Sets/2\)~is cartesian closed (by Exercise~12, and similar arguments).
\end{rmk}

\begin{rmk}
We have a functor \(\Sets^{(-)}:\Setsop\to\Cat\), which maps a set~\(I\) to the category~\(\Sets^I\) and maps a function \(f:J\to I\) to the functor \(\Sets^f:\Sets^I\to\Sets^J\). Indeed, note for \(g:K\to J\) that
\[\Sets^{fg}=\Sets^g\after\Sets^f\]
since
\begin{align*}
\Sets^{fg}((A_i)_{i\in I})&=(A_{(fg)(k)})_{k\in K}\\
	&=(A_{f(gk)})_{k\in K}\\
	&=\Sets^g((A_{f(j)})_{j\in J})\\
	&=\Sets^g(\Sets^f((A_i)_{i\in I}))\\
	&=(\Sets^g\after\Sets^f)((A_i)_{i\in I})
\end{align*}
\end{rmk}

\begin{rmk}
For a category~\(\C\) and functors \(F,G:\C\to\Cat\), say that \(\Phi:F\to G\) is a \emph{natural equivalence of categories}\footnote{A better concept for describing this phenomenon is that of \emph{pseudonatural equivalence} in 2-categories. See \url{https://math.stackexchange.com/q/3713074}.} if it is a family of equivalences of categories
\[(\Phi_C:FC\eqv GC)_{C\in\C}\tag{1}\]
such that for all \(f:C\to C'\), there is a natural isomorphism
\[\Phi_{C'}\after Ff\iso Gf\after\Phi_C\tag{2}\]
---that is, the following diagram commutes up to natural isomorphism:
\begin{diagram}
FC			&\rTo^{\Phi_C}		&GC\\
\dTo<{Ff}	&					&\dTo>{Gf}\\
FC'			&\rTo_{\Phi_{C'}}	&GC'
\end{diagram}
Note this definition generalizes that of natural isomorphism for category-valued functors by allowing equivalence instead of isomorphism in~(1) and allowing natural isomorphism instead of equality in~(2).

The previous exercise shows that there is a natural equivalence of this sort between the reindexing functor \(\Sets^{(-)}:\Setsop\to\Cat\) and the pullback functor \((-)^*:\Setsop\to\Cat\).\footnote{See the remark at the beginning of Chapter~5 above.}
\end{rmk}

% References
\begin{thebibliography}{0}
\bibitem{awodey10} Awodey, Steve. \textit{Category Theory, 2nd~ed.} Oxford, 2010.
\end{thebibliography}
\end{document}