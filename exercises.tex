% Notes and exercises on Category Theory
\documentclass[letterpaper,12pt]{article}
\usepackage{amsmath,amssymb,amsthm,enumitem,fourier}

\newcommand{\N}{\mathbb{N}}

\newcommand{\iso}{\cong}
\newcommand{\into}{\rightarrowtail}
\newcommand{\onto}{\twoheadrightarrow}

\newcommand{\after}{\circ}

\DeclareMathOperator{\dom}{dom}
\DeclareMathOperator{\cod}{cod}
\DeclareMathOperator{\fdom}{\mathbf{dom}}
\DeclareMathOperator{\fst}{fst}
\DeclareMathOperator{\snd}{snd}
\DeclareMathOperator{\Hom}{Hom}

\newcommand{\pair}[2]{\langle{#1},{#2}\rangle}
\newcommand{\copair}[2]{[{#1},{#2}]}
\newcommand{\comp}[1]{\overline{#1}}
\newcommand{\inv}[1]{#1^{-1}}
\renewcommand{\star}[1]{#1^{*}}
\newcommand{\cat}[1]{\mathbf{#1}}
\newcommand{\dual}[1]{#1^{\mathrm{op}}}
\newcommand{\arr}[1]{#1^{\rightarrow}}
\newcommand{\under}[1]{|{#1}|}

\newcommand{\C}{\cat{C}}
\newcommand{\Cop}{\dual{\C}}
\newcommand{\Rel}{\cat{Rel}}
\newcommand{\Sets}{\cat{Sets}}
\newcommand{\Mon}{\cat{Mon}}
\newcommand{\Pre}{\cat{Pre}}
\newcommand{\Pos}{\cat{Pos}}
\newcommand{\Types}{\C(\lambda)}
\newcommand{\Cat}{\cat{Cat}}

% Theorems
\theoremstyle{definition}
\newtheorem*{exer}{Exercise}

\theoremstyle{remark}
\newtheorem*{rmk}{Remark}

\newtheoremstyle{direction}{0.5em}{0.5em}{}{}{}{}{0.5em}{}
\theoremstyle{direction}
\newtheorem*{fwd}{\(\implies\)}
\newtheorem*{bwd}{\(\impliedby\)}

% Meta
\title{\textit{Category Theory}\\Notes and Exercises}
\author{John Peloquin}
\date{}

\begin{document}
\maketitle

\section*{Chapter 1}
\begin{exer}[1]
In~\(\Rel\), let the objects be sets and the arrows be relations between sets,\footnote{An arrow \(A\to B\) between sets \(A\)~and~\(B\) is understood as a triple \((R,A,B)\) with \(R\subseteq A\times B\).} with identities and composites defined as usual for relations.
\begin{enumerate}[itemsep=0pt]
\item[(a)] \(\Rel\)~is a category.
\item[(b)] Let \(G:\Sets\to\Rel\) map sets to themselves and functions to their graphs, so \(G(A)=A\) and
\[G(f:A\to B)=\{\,\pair{x}{f(x)}\mid x\in A\,\}\subseteq A\times B\]
Then \(G\)~is a functor.
\item[(c)] Let \(C:\dual{\Rel}\to\Rel\) map sets to themselves and relations to their inverses, so \(C(A)=A\) and
\[C(R\subseteq A\times B)=\inv{R}=\{\,\pair{y}{x}\mid\pair{x}{y}\in R\,\}\subseteq B\times A\]
Then \(C\)~is a functor.
\end{enumerate}
\begin{proof}\
\begin{enumerate}[itemsep=0pt]
\item[(a)] We must verify that composition of relations is associative and unital. Suppose \(R\subseteq A\times B\), \(S\subseteq B\times C\), and \(T\subseteq C\times D\). For \(\pair{w}{z}\in A\times D\), by the definition of composition we have
\begin{align*}
\pair{w}{z}\in(T\after S)\after R&\iff\exists x\in B[\pair{w}{x}\in R\land\pair{x}{z}\in T\after S]\\
								&\iff\exists x\in B,y\in C[\pair{w}{x}\in R\land\pair{x}{y}\in S\land\pair{y}{z}\in T]\\
								&\iff\exists y\in C[\pair{w}{y}\in S\after R\land\pair{y}{z}\in T]\\
								&\iff\pair{w}{z}\in T\after(S\after R)
\end{align*}
So \((T\after S)\after R=T\after(S\after R)\). It is immediate that \(R\after 1_A=R=1_B\after R\), where \(1_X\)~denotes the identity relation on~\(X\).
\item[(b)] By construction, \(G\)~maps objects to objects and arrows to arrows, and \(G(f:A\to B)\)~is an arrow from \(G(A)=A\) to \(G(B)=B\). Clearly \(G(1_A)=1_A=1_{G(A)}\). If \(f:A\to B\) and \(g:B\to C\), then \((g\after f)(x)=z\) if and only if \(f(x)=y\) and \(g(y)=z\), so \(G(g\after f)=G(g)\after G(f)\).
\item[(c)] Recall for a relation \(R\subseteq A\times B\), \(R\)~is represented as an arrow in~\(\Rel\) by the triple~\((R,A,B)\), and in~\(\dual{\Rel}\) by the triple~\((R,B,A)\), where it is denoted by~\(\star{R}\).\footnote{Importantly, \(R\)~is \emph{not} represented in~\(\dual{\Rel}\) by~\((\inv{R},B,A)\). The arrow is reversed by swapping the domain and codomain, but the underlying relation (set of ordered pairs) is unchanged.} So in~\(\Rel\), \(\dom R=A\) and \(\cod R=B\), whereas in~\(\dual{\Rel}\), \(\dom\star{R}=B=\star{B}\) and \(\cod\star{R}=A=\star{A}\), where \(R\)~and~\(\star{R}\) are here treated as arrows.

By construction, \(C\)~maps objects to objects and arrows to arrows. Now \(C((R,B,A))=(\inv{R},B,A)\), so \(C\)~preserves domains and codomains. Also
\[C(1_{\star{A}})=C(\star{1_A})=\inv{1_A}=1_A=1_{C(\star{A})}\]
For \(S\subseteq B\times C\),
\[C(\star{R}\after\star{S})=C(\star{(S\after R)})=\inv{(S\after R)}=\inv{R}\after\inv{S}=C(\star{R})\after C(\star{S})\qedhere\]
\end{enumerate}
\end{proof}
\end{exer}

\begin{exer}[2]\
\begin{enumerate}[itemsep=0pt]
\item[(a)] \(\Rel\iso\dual{\Rel}\)
\item[(c)] For any set~\(X\) with powerset~\(P(X)\), \(P(X)\iso\dual{P(X)}\) as poset categories.
\end{enumerate}
\begin{proof}\
\begin{enumerate}[itemsep=0pt]
\item[(a)] The functor in Exercise~1(c) is its own inverse, hence is an isomorphism.
\item[(c)] Recall in~\(P(X)\) there exists a unique arrow \(A\to B\) if and only if \(A\subseteq B\), hence in~\(\dual{P(X)}\) there exists a unique arrow \(A\to B\) if and only if \(A\supseteq B\).

For \(A\subseteq X\), write \(\comp{A}=X-A=\{\,x\in X\mid x\not\in A\,\}\). Define \(C:\dual{P(X)}\to P\) by \(C(A)=\comp{A}\) and
\[C(A\to B)=\comp{A}\to\comp{B}=C(A)\to C(B)\]
which is well defined since \(A\supseteq B\) if and only if \(\comp{A}\subseteq\comp{B}\). Clearly \(C\)~maps objects to objects and arrows to arrows, and also preserves domains and codomains. Substituting \(A\)~for~\(B\) above shows that \(C\)~preserves identities. For \(X\supseteq A\supseteq B\supseteq C\),
\[C(A\to B\to C)=\comp{A}\to\comp{B}\to\comp{C}=C(A)\to C(B)\to C(C)\]
so \(C\)~preserves composites. Therefore \(C\)~is a functor. Since \(C\)~is clearly its own inverse, \(C\)~is an isomorphism.\qedhere
\end{enumerate}
\end{proof}
\end{exer}

\begin{exer}[3]\
\begin{enumerate}[itemsep=0pt]
\item[(a)] In~\(\Sets\), the isomorphisms are precisely the bijections.
\item[(b)] In~\(\Mon\), the isomorphisms are precisely the bijective homomorphisms.
\item[(c)] In~\(\Pos\), the isomorphisms are \emph{not} the bijective homomorphisms.
\end{enumerate}
\begin{proof}\
\begin{enumerate}
\item[(a)] A function \(f:A\to B\) has a (two-sided) inverse if and only if it is bijective. Indeed, suppose \(g:B\to A\) is an inverse of~\(f\). If \(a,a'\in A\) and \(f(a)=f(a')\), then
\[a=1_A(a)=(g\after f)(a)=g(f(a))=g(f(a'))=(g\after f)(a')=1_A(a')=a'\]
If \(b\in B\), then \(b=1_B(b)=(f\after g)(b)=f(g(b))\). Conversely, if \(f\)~is bijective, then for each \(b\in B\) we can let~\(g(b)\) to be the unique \(a\in A\) with \(f(a)=b\). Then \(g:B\to A\) is clearly an inverse of~\(f\).
\item[(b)] A monoid homomorphism is, in particular, a function, hence an isomorphism is a bijective homomorphism by~(a). Conversely, if \(f:A\to B\) is a bijective homomorphism, then \(f\)~has an inverse \emph{function} \(g:B\to A\) by~(a). If \(b,b'\in B\), then
\[bb'=1_B(b)1_B(b')=(f\after g)(b)(f\after g)(b')=f(g(b))f(g(b'))=f(g(b)g(b'))\]
so
\[g(bb')=g(f(g(b)g(b')))=(g\after f)(g(b)g(b'))=1_A(g(b)g(b'))=g(b)g(b')\]
Therefore \(g\)~is a homomorphism and hence \(f\)~is an isomorphism.
\item[(c)] As in~(b), a poset homomorphism is, in particular, a function, hence an isomorphism is a bijective homomorphism by~(a). However, unlike in~(b), the inverse of a bijective homomorphism need not be a homomorphism. For example, consider a poset consisting of two copies of \(\N=(N,\le)\) with no relations between the copies. Map this poset into~\(\N\) by ``zipping'' the two copies together, sending one to the evens in order, and the other to the odds in order. This mapping is clearly a bijective homomorphism, but its inverse is not since, for example, \(0\le 1\) in the image, but the preimage of~\(0\) is not related to the preimage of~\(1\).\qedhere
\end{enumerate}
\end{proof}
\end{exer}

\begin{exer}[5]
Let \(\C\)~be a category and \(C\in\C\). Let \(U:\C/C\to\C\) ``forget about the base object~\(C\)'' by mapping each object \(f:A\to C\) to its domain~\(A\) and each arrow \(a:A\to B\) to ``itself.''\footnote{An arrow in~\(\C/C\) is understood as a triple \((a,f,f')\) where \(a:A\to B\), \(f:A\to C\), and \(f':B\to C\) are arrows in~\(\C\) with \(f=f'\after a\). So \(U((a,f,f'))=a\).} Then \(U\)~is a functor.

Let \(F:\C/C\to\arr{\C}\) map objects to themselves and each arrow \(a:A\to B\) to the pair~\((a,1_C)\), where \(1_C\)~is the identity arrow for~\(C\) in~\(\C\). Then \(F\)~is a functor, and \(\fdom\after F=U\), where \(\fdom:\arr{\C}\to\C\) is the functor mapping each object \(f:A\to B\) to its domain~\(A\) and each arrow \((g_1,g_2)\) to~\(g_1\).
\end{exer}
\begin{proof}
\(U\)~maps objects to objects and arrows to arrows, and preserves domains and codomains of arrows. Since \(\C/C\) inherits identities and composites from~\(\C\), \(U\)~also preserves identities and composites. Therefore \(U\)~is a functor.

\(F\)~maps objects to objects and arrows to arrows, and preserves domains and codomains of arrows, since if \(a:A\to B\) maps \(f:A\to C\) to \(f':B\to C\) in~\(\C/C\), then \(1_C\after f=f=f'\after a\), hence \((a,1_C)\)~maps~\(f\) to~\(f'\) in~\(\arr{\C}\). Since \(\arr{\C}\)~also inherits identities and composites from~\(\C\), \(F\)~also preserves identities and composites. Therefore \(F\)~is a functor. Clearly \(\fdom\after F=U\).
\end{proof}

\begin{exer}[6]
Let \(\C\)~be a category and \(C\in\C\). Then \(C/\C=\Cop/C\).\footnote{In this exercise, we informally identify an arrow \(f:A\to B\) in~\(\C\) with the corresponding reversed arrow \(\star{f}:\star{B}\to\star{A}\) in~\(\Cop\), and with any corresponding arrows in slices or coslices.}
\end{exer}
\begin{proof}
The arrows out of~\(C\) in~\(\C\) are precisely the arrows into~\(C\) in~\(\Cop\), and the commuting triangles among the former in~\(\C\) are precisely the commuting triangles among the latter in~\(\Cop\). Therefore \(C/\C\)~and~\(\Cop/C\) have the same objects and arrows. They also have the same identities and composites since these are inherited from \(\C\)~and~\(\Cop\), respectively, where they are the same by definition of~\(\Cop\).
\end{proof}
\begin{rmk}
Similarly \(C/\Cop=\C/C\).
\end{rmk}

\begin{exer}[8]
For a (small) category~\(\C\), let \(P(\C)\) consist of the objects from~\(\C\) ordered as follows:
\begin{center}
\(A\le B\) if and only if there exists an arrow \(A\to B\) in~\(\C\)
\end{center}
Then \(P(\C)\)~is a preorder, and \(P:\Cat\to\Pre\) determines a functor with \(P\after C=1_{\Pre}\), where \(C:\Pre\to\Cat\) is the evident inclusion functor.
\end{exer}
\begin{proof}
Reflexivity and transitivity of the order in~\(P(\C)\) follow from the existence of identities and composites in~\(\C\). So \(P\)~maps categories to preorders. For a functor \(F:\C\to\cat{D}\), let \(P(F)\)~be the restriction of~\(F\) to objects. If \(A\le B\) in~\(P(\C)\), there exists an arrow \(f:A\to B\) in~\(\C\). But then \(F(f):F(A)\to F(B)\) is an arrow in~\(\cat{D}\), so \(F(A)\le F(B)\) in~\(P(\cat{D})\). Therefore \(P(F)\)~is monotone, and hence \(P\)~maps functors to preorder homomorphisms. Since \(P\)~just restricts functors to objects, it preserves domains and codomains, identities, and composites, hence it is a functor. It is obvious that \(P\after C=1_{\Pre}\).
\end{proof}
\begin{rmk}
In general \(C\after P\ne 1_{\Cat}\) because \(P\)~loses information about the arrow structure of categories. Specifically, multiple arrows from one object to another will be represented by a single relation between those objects under~\(P\).
\end{rmk}

\begin{exer}[11]
There exists a functor \(M:\Sets\to\Mon\) mapping each set~\(A\) to the free monoid on~\(A\).
\end{exer}
\begin{proof}
We prove this in two ways.
\begin{enumerate}[itemsep=0pt]
\item[(a)] Let \(M(A)=\star{A}\) and for \(f:A\to B\) define \(M(f):\star{A}\to\star{B}\) by
\[M(f)(a_1\cdots a_k)=f(a_1)\cdots f(a_k)\quad a_1,\ldots,a_k\in A\]
\(M(f)\)~is well defined on~\(\star{A}\) since every element in~\(\star{A}\) can be expressed uniquely as a product of elements of~\(A\), and by construction \(M(f)\)~is a monoid homomorphism extending~\(f\). So \(M\)~maps objects to objects and arrows to arrows. Clearly \(M\)~preserves domains and codomains of arrows and \(M(1_A)=1_{\star{A}}\). If \(g:B\to C\), then
\begin{align*}
M(g\after f)(a_1\cdots a_k)&=(g\after f)(a_1)\cdots(g\after f)(a_k)\\
	&=g(f(a_1))\cdots g(f(a_k))\\
	&=M(g)(f(a_1)\cdots f(a_k))\\
	&=M(g)(M(f)(a_1\cdots a_k))\\
	&=(M(g)\after M(f))(a_1\cdots a_k)
\end{align*}
So \(M(g\after f)=M(g)\after M(f)\) and \(M\)~preserves composites. Therefore \(M\)~is a functor.
\item[(b)] Let \(M(A)\)~be ``the'' free monoid on~\(A\) satisfying the universal mapping property (Propositions 1.9~and~1.10). For \(f:A\to B\to\under{M(B)}\), let \(M(f)\)~be the unique monoid homomorphism from~\(M(A)\) to~\(M(B)\) extending~\(f\). Clearly \(M\)~maps objects to objects and arrows to arrows, and preserves domains and codomains of arrows. Now \(1_{M(A)}\)~extends~\(1_A\), hence we must have \(M(1_A)=1_{M(A)}\). Similarly if \(g:B\to C\to\under{M(C)}\), then \(M(g)\after M(f)\) extends~\(g\after f\), hence we must have \(M(g\after f)=M(g)\after M(f)\). Therefore \(M\)~is a functor.\qedhere
\end{enumerate}
\end{proof}
\begin{rmk}
A homomorphism \(h:M(A)\to B\) is uniquely determined by its action on~\(A\), where this action is \(\under{h}\after i:A\to\under{M(A)}\to\under{B}\). This is trivially true by the universal mapping property since \(h\)~extends~\(\under{h}\after i\) to~\(M(A)\), that is, \(\under{h}\after i=\under{h}\after i\). This is a familiar concept in mathematics (for example, a linear transformation of a vector space is uniquely determined by its action on a basis, etc.).
\end{rmk}

\section*{Chapter~2}
\begin{exer}[1]
In~\(\Sets\), the epis are precisely the surjections. Therefore the isos are precisely the epi-monos.
\end{exer}
\begin{proof}
If \(f:A\to B\) is a surjection, then \(f\)~has a right inverse (AC), hence \(f\)~is a split epi. Conversely, if \(f\)~is not a surjection, there exists \(b\in B\) with \(b\not\in f[A]\). Define \(g:B\to 2\) by
\[g(x)=\begin{cases}
1&\text{if }x=b\\
0&\text{otherwise}
\end{cases}\]
Then \(g\ne0\), but \(g\after f=0\after f\), so \(f\)~is not an epi. Therefore the epis are precisely the surjections.

Now by this result and Proposition~2.2, the epi-monos are precisely the bijections. By Exercise~1.3, the bijections are precisely the isos. Therefore the epi-monos are precisely the isos.
\end{proof}

\begin{exer}[2]
In a poset category, every arrow is an epi-mono since there is at most one arrow between any two objects.
\end{exer}

\begin{exer}[3]
Inverses are unique.
\end{exer}
\begin{proof}
If \(f:A\to B\) and \(g,g':B\to A\) are inverses of~\(f\), then
\[g=g\after 1_B=g\after(f\after g')=(g\after f)\after g'=1_A\after g'=g'\qedhere\]
\end{proof}

\begin{exer}[4]
Let \(f:A\to B\), \(g:B\to C\), and \(h:A\to C\) form a commutative triangle (\(h=g\after f\)).
\begin{enumerate}[itemsep=0pt]
\item[(a)] If \(f\)~and~\(g\) are monic [epic, iso], so is~\(h\).
\item[(b)] If \(h\)~is monic, so is~\(f\).
\item[(c)] If \(h\)~is epic, so is~\(g\).
\item[(d)] If \(h\)~is monic, \(g\)~need not be.
\item[(e)] If \(h\)~is epic, \(f\)~need not be.
\end{enumerate}
\begin{proof}\
\begin{enumerate}[itemsep=0pt]
\item[(a)] Suppose \(f\)~and~\(g\) are monic. If \(x,y:D\to A\) and \(h\after x=h\after y\), then
\[g\after(f\after x)=(g\after f)\after x=h\after x=h\after y=(g\after f)\after y=g\after(f\after y)\]
so \(f\after x=f\after y\) since \(g\)~is monic, and \(x=y\) since \(f\)~is monic. Therefore \(h\)~is monic.

Suppose \(f\)~and~\(g\) are epic. If \(i,j:C\to D\) and \(i\after h=j\after h\), then
\[(i\after g)\after f=i\after (g\after f)=i\after h=j\after h=j\after(g\after f)=(j\after g)\after f\]
so \(i\after g=j\after g\) since \(f\)~is epic, and \(i=j\) since \(g\)~is epic. Therefore \(h\)~is epic.\footnote{This also follows from the previous result by duality. If \(f\)~and~\(g\) are epic in~\(\C\), then \(\star{f}\)~and~\(\star{g}\) are monic in~\(\Cop\), so \(\star{h}\)~is monic in~\(\Cop\), so \(h\)~is epic in~\(\C\).}

If \(f\)~and~\(g\) are isos, then \(\inv{h}=\inv{f}\after\inv{g}\), so \(h\)~is an iso.
\item[(b)] If \(f\)~is not monic, choose \(x\ne y\) such that \(f\after x=f\after y\). Then
\[h\after x=(g\after f)\after x=g\after(f\after x)=g\after(f\after y)=(g\after f)\after y=h\after y\]
So \(h\)~is not monic.
\item[(c)] If \(g\)~is not epic, choose \(i\ne j\) such that \(i\after g=j\after g\). Then
\[i\after h=i\after(g\after f)=(i\after g)\after f=(j\after g)\after f=j\after(g\after f)=j\after h\]
So \(h\)~is not epic.\footnote{This also follows from the previous result by duality. If \(h\)~is epic in~\(\C\), then \(\star{h}=\star{f}\after\star{g}\)~is monic in~\(\Cop\), so \(\star{g}\)~is monic in~\(\Cop\), so \(g\)~is epic in~\(\C\).}
\item[(d),(e)] In~\(\Sets\), let \(A=C=1\) and \(B=2\) and let \(f=0_{A\to B}\) and \(g=0_{B\to C}\). Then \(h=0_{A\to C}\) is both monic and epic, but \(g\)~is not monic and \(f\)~is not epic.\qedhere
\end{enumerate}
\end{proof}
\end{exer}

\begin{exer}[5]
For \(f:A\to B\), the following are equivalent:
\begin{enumerate}[itemsep=0pt]
\item[(a)] \(f\)~is an iso.
\item[(b)] \(f\)~is a mono and a split epi.
\item[(c)] \(f\)~is a split mono and an epi.
\item[(d)] \(f\)~is a split mono and a split epi.
\end{enumerate}
\begin{proof}
It is immediate that (a)~\(\implies\)~(d)~\(\implies\)~(b),(c).

For (b)~\(\implies\)~(a), suppose that \(f\)~is monic and \(g:B\to A\) satisfies \(f\after g=1_B\). We claim \(g\)~also satisfies \(g\after f=1_A\), so \(f\)~is an iso. But this follows from
\[f\after(g\after f)=(f\after g)\after f=1_B\after f=f=f\after 1_A\]
since \(f\)~is monic.

For (c)~\(\implies\)~(a), suppose that \(f\)~is epic and \(g:B\to A\) satisfies \(g\after f=1_A\). We claim \(g\)~also satisfies \(f\after g=1_B\), so \(f\)~is an iso. But this follows from
\[(f\after g)\after f=f\after(g\after f)=f\after 1_A=f=1_B\after f\]
since \(f\)~is epic.\footnote{This also follows from the previous result by duality. If \(f\)~is epic and \(g\)~is a left inverse of~\(f\) in~\(\C\), then \(\star{f}\)~is monic and \(\star{g}\)~is a right inverse of~\(\star{f}\) in~\(\Cop\). Therefore \(\star{f}\)~is an iso in~\(\Cop\), so \(f\)~is an iso in~\(\C\).} 
\end{proof}
\end{exer}

\begin{exer}[7]
A retract of a projective object is projective.
\end{exer}
\begin{proof}
Let \(P\)~be projective and \(R\)~be a retract of~\(P\) where \(s:R\to P\), \(r:P\to R\), and \(r\after s=1_R\). Suppose \(f:R\to Y\) and \(e:X\onto Y\). Note \(f\after r:P\to Y\), so by projectivity of~\(P\) there exists \(p:P\to X\) such that \(e\after p=f\after r\). Now \(p\after s:R\to X\) and
\[e\after(p\after s)=(e\after p)\after s=(f\after r)\after s=f\after(r\after s)=f\after 1_R=f\]
Therefore \(R\)~is projective.
\end{proof}

\begin{exer}[8]
In~\(\Sets\), every set is projective.
\end{exer}
\begin{proof}
If \(f:P\to Y\) and \(g:X\onto Y\), then since \(g\)~is surjective (Exercise~1), \(g\)~has a right inverse \(h:Y\to X\) with \(g\after h=1_Y\). Set \(p=h\after f:P\to X\). Then
\[g\after p=g\after(h\after f)=(g\after h)\after f=1_Y\after f=f\]
Therefore \(P\)~is projective. 
\end{proof}
\begin{rmk}
Projectivity is more interesting in categories of structured sets, where it implies ``freeness'' of structure allowing factoring of outgoing homomorphisms.
\end{rmk}

\begin{exer}[11]
For a set~\(A\), let \(A\text{-}\Mon\) be the category of \(A\)-monoids \((M,m)\), where \(M\)~is a monoid and \(m:A\to U(M)\), with arrows \(h:(M,m)\to(N,n)\), where \(h:M\to N\) is a monoid homomorphism and \(n=U(h)\after m\).

An initial object in \(A\text{-}\Mon\) is just a free monoid on~\(A\) in~\(\Mon\).
\end{exer}
\begin{proof}
The \(A\)-monoid~\((M,m)\) is initial if and only if for all \(A\)-monoids~\((N,n)\), there is a unique \(A\)-monoid homomorphism \(h:(M,m)\to(N,n)\). This is just to say that \(m:A\to U(M)\) and for all monoids~\(N\) with \(n:A\to U(N)\) there is a unique monoid homomorphism \(h:M\to N\) with \(n=U(h)\after m\). But this is just the universal mapping property for the free monoid on~\(A\) in~\(\Mon\).
\end{proof}

\begin{exer}[15]
For a category~\(\C\) and objects \(A,B\in\C\), let \(\C_{A,B}\)~be the category with objects \((X,x_1,x_2)\), where \(x_1:X\to A\) and \(x_2:X\to B\) in~\(\C\), and with arrows \(f:(X,x_1,x_2)\to(Y,y_1,y_2)\), where \(f:X\to Y\) and \(x_i=y_i\after f\) in~\(\C\).

A terminal object in~\(\C_{A,B}\) is just a product of \(A\)~and~\(B\) in~\(\C\).
\end{exer}
\begin{proof}
Object \((P,p_1,p_2)\) in~\(\C_{A,B}\) is terminal if and only if for all objects \((X,x_1,x_2)\) there is a unique \(p:(X,x_1,x_2)\to(P,p_1,p_2)\). This is just to say that \(p_1:P\to A\), \(p_2:P\to B\), and for all objects \(X\in\C\) with arrows \(x_1:X\to A\) and \(x_2:X\to B\) there is a unique \(p:X\to P\) with \(x_i=p_i\after p\). But this is just the universal mapping property for the product~\(A\times B\) in~\(\C\).
\end{proof}
\begin{rmk}
The objects in~\(\C_{A,B}\) are just pairs of ``generalized elements'' of \(A\)~and~\(B\) in~\(\C\). A terminal object in~\(\C_{A,B}\) has a unique ``generalized element'' for every such pair, hence it is just the product~\(A\times B\).
\end{rmk}

\begin{exer}[16]
Let \(\Types\)~be the category of types in the \(\lambda\)-calculus. Then the product functor \(\times:\Types\times\Types\to\Types\) maps objects \(A\)~and~\(B\) to~\(A\times B\) and arrows \(f:A\to B\) and \(g:A'\to B'\) to \(f\times g:A\times A'\to B\times B'\) where
\[f\times g=\lambda c.\pair{f(\fst(c))}{g(\snd(c))}\]
For any fixed type~\(A\), there is a functor~\(A\to(-)\) on~\(\Types\) taking each type~\(X\) to the type~\(A\to X\).
\end{exer}
\begin{proof}
We know that \(\Types\)~has products, so the product functor is defined on~\(\Types\). For \(f:A\to B\) and \(g:A'\to B\), if
\[A\xleftarrow{p_1}A\times A'\xrightarrow{p_2}A'\]
where \(p_1=\lambda z.\fst(z)\) and \(p_2=\lambda z.\snd(z)\), then
\begin{align*}
f\times g&=\pair{f\after p_1}{g\after p_2}\\
	&=\lambda c.\pair{f(p_1c)}{g(p_2c)}\\
	&=\lambda c.\pair{f(\fst(c))}{g(\snd(c))}
\end{align*}
Fix a type~\(A\). Let \(A\to(-)\)~map each type~\(X\) to the type~\(A\to X\) and map each function \(f:X\to Y\) to the function \(\overline{f}:(A\to X)\to(A\to Y)\) given by \(\overline{f}=\lambda g.f\after g\), where \(f\after g=\lambda x.f(gx)\). We claim this mapping is a functor.

Indeed, this mapping clearly maps objects to objects and arrows to arrows and it preserves domains and codomains of arrows. It also clearly preserves identities. If \(g:Y\to Z\), then
\begin{align*}
\overline{g\after f}&=\lambda h.(g\after f)\after h\\
	&=\lambda h.g\after(f\after h)\\
	&=\lambda h.\overline{g}(\overline{f}h)\\
	&=\overline{g}\after\overline{f}
\end{align*}
So the mapping also preserves composites.\footnote{Note preservation of identities and composites relies on \(\beta\eta\)-equivalence for \emph{equality} of the functions involved.} Therefore it is a functor.
\end{proof}
\begin{rmk}
This result shows that in functional programming languages such as Haskell, functions of a fixed input type are ``functorial'' types. This implies that functions on arbitrary types can be lifted to operate on such functions through composition.
\end{rmk}

\begin{exer}[17]
In any category~\(\C\) with products, define the \emph{graph} of an arrow \(f:A\to B\) by
\[\Gamma(f)=\pair{1_A}{f}:A\into A\times B\]
Then \(\Gamma(f)\)~is a mono for every arrow~\(f\).

In~\(\Sets\), \(\Gamma\)~determines a functor \(G:\Sets\to\Rel\) mapping sets to themselves and functions to their graphs.
\end{exer}
\begin{proof}
To see that \(\Gamma(f)\)~is a mono, suppose \(x,y:X\to A\) and \(\Gamma(f)\after x=\Gamma(f)\after y\). By the universal mapping property of~\(A\times B\),
\[\Gamma(f)\after x=\pair{1_A}{f}\after x=\pair{1_A\after x}{f\after x}=\pair{x}{f\after x}\]
Similarly \(\Gamma(f)\after y=\pair{y}{f\after y}\). Therefore \(\pair{x}{f\after x}=\pair{y}{f\after y}\), so \(x=y\).

Define \(G:\Sets\to\Rel\) by \(G(A)=A\) and \(G(f:A\to B)=\Gamma(f)[A]\subseteq A\times B\). Then clearly \(G\)~is just the map from Exercise~1.1(b), which is a functor.
\end{proof}

\section*{Chapter~3}
\begin{exer}[1]
Let \(\C\)~be a (locally small) category. Then
\[A\xrightarrow{c_1}C\xleftarrow{c_2}B\]
is a coproduct if and only if for all objects~\(Z\) the function
\begin{align*}
\Hom(C,Z)&\to\Hom(A,Z)\times\Hom(B,Z)\\
f&\mapsto\pair{f\after c_1}{f\after c_2}
\end{align*}
is an iso.
\end{exer}
\begin{proof}
By duality, the given diagram is a coproduct in~\(\C\) if and only if
\[\star{A}\xleftarrow{\star{c_1}}\star{C}\xrightarrow{\star{c_2}}\star{B}\]
is a product in~\(\Cop\). We know this diagram is a product in~\(\Cop\) if and only if for all objects~\(\star{Z}\), the function
\begin{align*}
\Hom_{\Cop}(\star{Z},\star{C})&\to\Hom_{\Cop}(\star{Z},\star{A})\times\Hom_{\Cop}(\star{Z},\star{B})\\
\star{f}&\mapsto\pair{\star{c_1}\after\star{f}}{\star{c_2}\after\star{f}}
\end{align*}
is an iso (Proposition~2.20). But by definition of~\(\Cop\),
\[\Hom_{\Cop}(\star{Z},\star{X})=\Hom_{\C}(X,Z)\]
for all objects \(X\in\C\) and \(\star{c_i}\after\star{f}=\star{(f\after c_i)}\) for all arrows \(f\in\C\), so the functions are the same.
\end{proof}

\begin{exer}[2]
The free monoid functor preserves coproducts, that is,
\[M(A+B)\iso M(A)+M(B)\]
\end{exer}
\begin{proof}
By the universal property of~\(M(A)\), let \(i_1:M(A)\to M(A+B)\) extend the inclusion \(A\to A+B\to\under{M(A+B)}\).\footnote{The inclusion \(A\to A+B\) is from the coproduct construction, and the inclusion \(A+B\to\under{M(A+B)}\) is from the free monoid construction.} Similarly let \(i_2:M(B)\to M(A+B)\) extend the inclusion \(B\to A+B\to\under{M(A+B)}\). We claim
\[M(A)\xrightarrow{i_1}M(A+B)\xleftarrow{i_2}M(B)\]
is a coproduct of \(M(A)\)~and~\(M(B)\), from which the desired result follows by uniqueness of the coproduct (Proposition~3.12).

Given \(x:M(A)\to N\) and \(y:M(B)\to N\), let \(\under{x}_A:A\to\under{N}\) be the composite of the inclusion \(A\to\under{M(A)}\) with \(\under{x}:\under{M(A)}\to\under{N}\), and similarly let \(\under{y}_B:B\to\under{N}\) be the composite of the inclusion \(B\to\under{M(B)}\) with \(\under{y}:\under{M(B)}\to\under{N}\). By the universal property of~\(A+B\), there is a unique copairing~\(\copair{\under{x}_A}{\under{y}_B}:A+B\to\under{N}\), and by the universal property of~\(M(A+B)\) this copairing extends uniquely to a homomorphism \(z:M(A+B)\to N\).

It follows from the copairing and the universal property of~\(M(A)\) that \(z\after i_1=x\) since both \(z\after i_1\)~and~\(x\) extend~\(\under{x}_A\) to~\(M(A)\), and similarly \(z\after i_2=y\). Moreover it follows from uniqueness of the copairing and the extension that \(z\)~uniquely satisfies these equations. Therefore \(z\)~is a copairing~\(\copair{x}{y}\), and \(M(A+B)\)~is a coproduct of \(M(A)\)~and~\(M(B)\) as claimed.
\end{proof}

% References
\begin{thebibliography}{0}
\bibitem{awodey10} Awodey, Steve. \textit{Category Theory, 2nd~ed.} Oxford, 2010.
\end{thebibliography}
\end{document}