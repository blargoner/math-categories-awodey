% Notes and exercises on Category Theory
\documentclass[letterpaper,12pt]{article}
\usepackage{amsmath,amssymb,amsthm,enumitem,fourier}

\newcommand{\N}{\mathbb{N}}

\newcommand{\Rel}{\mathbf{Rel}}
\newcommand{\Sets}{\mathbf{Sets}}
\newcommand{\Mon}{\mathbf{Mon}}
\newcommand{\Pre}{\mathbf{Pre}}
\newcommand{\Pos}{\mathbf{Pos}}
\newcommand{\Cat}{\mathbf{Cat}}

\newcommand{\iso}{\cong}

\newcommand{\after}{\circ}

\DeclareMathOperator{\dom}{dom}
\DeclareMathOperator{\cod}{cod}
\DeclareMathOperator{\fdom}{\mathbf{dom}}

\newcommand{\pair}[2]{\langle{#1},{#2}\rangle}
\newcommand{\comp}[1]{\overline{#1}}
\renewcommand{\star}[1]{#1^{*}}
\newcommand{\cat}[1]{\mathbf{#1}}
\newcommand{\dual}[1]{#1^{\mathrm{op}}}
\newcommand{\arr}[1]{#1^{\rightarrow}}

% Theorems
\theoremstyle{definition}
\newtheorem*{exer}{Exercise}

\theoremstyle{remark}
\newtheorem*{rmk}{Remark}

\newtheoremstyle{direction}{0.5em}{0.5em}{}{}{}{}{0.5em}{}
\theoremstyle{direction}
\newtheorem*{fwd}{\(\implies\)}
\newtheorem*{bwd}{\(\impliedby\)}

% Meta
\title{\textit{Category Theory}\\Notes and Exercises}
\author{John Peloquin}
\date{}

\begin{document}
\maketitle

\section*{Chapter 1}
\begin{exer}[1]
In~\(\Rel\), let the objects be sets and the arrows be relations between sets,\footnote{An arrow \(A\to B\) between sets \(A\)~and~\(B\) is understood as a triple \((R,A,B)\) with \(R\subseteq A\times B\).} with identities and composites defined as usual for relations.
\begin{enumerate}[itemsep=0pt]
\item[(a)] \(\Rel\)~is a category.
\item[(b)] Let \(G:\Sets\to\Rel\) map sets to themselves and functions to their graphs, so \(G(A)=A\) and
\[G(f:A\to B)=\{\,\pair{x}{f(x)}\mid x\in A\,\}\subseteq A\times B\]
Then \(G\)~is a functor.
\item[(c)] Let \(C:\dual{\Rel}\to\Rel\) map sets to themselves and relations to their inverses, so \(C(A)=A\) and
\[C(R\subseteq A\times B)=R^{-1}=\{\,\pair{y}{x}\mid\pair{x}{y}\in R\,\}\subseteq B\times A\]
Then \(C\)~is a functor.
\end{enumerate}
\begin{proof}\
\begin{enumerate}[itemsep=0pt]
\item[(a)] We must verify that composition of relations is associative and unital. Suppose \(R\subseteq A\times B\), \(S\subseteq B\times C\), and \(T\subseteq C\times D\). For \(\pair{w}{z}\in A\times D\), by the definition of composition we have
\begin{align*}
\pair{w}{z}\in(T\after S)\after R&\iff\exists x\in B[\pair{w}{x}\in R\land\pair{x}{z}\in T\after S]\\
								&\iff\exists x\in B,y\in C[\pair{w}{x}\in R\land\pair{x}{y}\in S\land\pair{y}{z}\in T]\\
								&\iff\exists y\in C[\pair{w}{y}\in S\after R\land\pair{y}{z}\in T]\\
								&\iff\pair{w}{z}\in T\after(S\after R)
\end{align*}
So \((T\after S)\after R=T\after(S\after R)\). It is immediate that \(R\after 1_A=R=1_B\after R\), where \(1_X\)~denotes the identity relation on~\(X\).
\item[(b)] By construction, \(G\)~maps objects to objects and arrows to arrows, and \(G(f:A\to B)\)~is an arrow from \(G(A)=A\) to \(G(B)=B\). Clearly \(G(1_A)=1_A=1_{G(A)}\). If \(f:A\to B\) and \(g:B\to C\), then \((g\after f)(x)=z\) if and only if \(f(x)=y\) and \(g(y)=z\), so \(G(g\after f)=G(g)\after G(f)\).
\item[(c)] Recall for a relation \(R\subseteq A\times B\), \(R\)~is represented as an arrow in~\(\Rel\) by the triple~\((R,A,B)\), and in~\(\dual{\Rel}\) by the triple~\((R,B,A)\), where it is denoted by~\(\star{R}\).\footnote{Importantly, \(R\)~is \emph{not} represented in~\(\dual{\Rel}\) by~\((R^{-1},B,A)\). The arrow is reversed by swapping the domain and codomain, but the underlying relation (set of ordered pairs) is unchanged.} So in~\(\Rel\), \(\dom R=A\) and \(\cod R=B\), whereas in~\(\dual{\Rel}\), \(\dom\star{R}=B=\star{B}\) and \(\cod\star{R}=A=\star{A}\), where \(R\)~and~\(\star{R}\) are here treated as arrows.

By construction, \(C\)~maps objects to objects and arrows to arrows. Now \(C((R,B,A))=(R^{-1},B,A)\), so \(C\)~preserves domains and codomains. Also
\[C(1_{\star{A}})=C(\star{1_A})=1_A^{-1}=1_A=1_{C(\star{A})}\]
For \(S\subseteq B\times C\),
\[C(\star{R}\after\star{S})=C(\star{(S\after R)})=(S\after R)^{-1}=R^{-1}\after S^{-1}=C(\star{R})\after C(\star{S})\qedhere\]
\end{enumerate}
\end{proof}
\end{exer}

\begin{exer}[2]\
\begin{enumerate}[itemsep=0pt]
\item[(a)] \(\Rel\iso\dual{\Rel}\)
\item[(c)] For any set~\(X\) with powerset~\(P(X)\), \(P(X)\iso\dual{P(X)}\) as poset categories.
\end{enumerate}
\begin{proof}\
\begin{enumerate}[itemsep=0pt]
\item[(a)] The functor in Exercise~1(c) is its own inverse, hence is an isomorphism.
\item[(c)] Recall in~\(P(X)\) there exists a unique arrow \(A\to B\) if and only if \(A\subseteq B\), hence in~\(\dual{P(X)}\) there exists a unique arrow \(A\to B\) if and only if \(A\supseteq B\).

For \(A\subseteq X\), write \(\comp{A}=X-A=\{\,x\in X\mid x\not\in A\,\}\). Define \(C:\dual{P(X)}\to P\) by \(C(A)=\comp{A}\) and
\[C(A\to B)=\comp{A}\to\comp{B}=C(A)\to C(B)\]
which is well defined since \(A\supseteq B\) if and only if \(\comp{A}\subseteq\comp{B}\). Clearly \(C\)~maps objects to objects and arrows to arrows, and also preserves domains and codomains. Substituting \(A\)~for~\(B\) above shows that \(C\)~preserves identities. For \(X\supseteq A\supseteq B\supseteq C\),
\[C(A\to B\to C)=\comp{A}\to\comp{B}\to\comp{C}=C(A)\to C(B)\to C(C)\]
so \(C\)~preserves composites. Therefore \(C\)~is a functor. Since \(C\)~is clearly its own inverse, \(C\)~is an isomorphism.\qedhere
\end{enumerate}
\end{proof}
\end{exer}

\begin{exer}[3]\
\begin{enumerate}[itemsep=0pt]
\item[(a)] In~\(\Sets\), the isomorphisms are precisely the bijections.
\item[(b)] In~\(\Mon\), the isomorphisms are precisely the bijective homomorphisms.
\item[(c)] In~\(\Pos\), the isomorphisms are \emph{not} the bijective homomorphisms.
\end{enumerate}
\begin{proof}\
\begin{enumerate}
\item[(a)] A function \(f:A\to B\) has a (two-sided) inverse if and only if it is bijective. Indeed, suppose \(g:B\to A\) is an inverse of~\(f\). If \(a,a'\in A\) and \(f(a)=f(a')\), then
\[a=1_A(a)=(g\after f)(a)=g(f(a))=g(f(a'))=(g\after f)(a')=1_A(a')=a'\]
If \(b\in B\), then \(b=1_B(b)=(f\after g)(b)=f(g(b))\). Conversely, if \(f\)~is bijective, then for each \(b\in B\) we can let~\(g(b)\) to be the unique \(a\in A\) with \(f(a)=b\). Then \(g:B\to A\) is clearly an inverse of~\(f\).
\item[(b)] A monoid homomorphism is, in particular, a function, hence an isomorphism is a bijective homomorphism by~(a). Conversely, if \(f:A\to B\) is a bijective homomorphism, then \(f\)~has an inverse \emph{function} \(g:B\to A\) by~(a). If \(b,b'\in B\), then
\[bb'=1_B(b)1_B(b')=(f\after g)(b)(f\after g)(b')=f(g(b))f(g(b'))=f(g(b)g(b'))\]
so
\[g(bb')=g(f(g(b)g(b')))=(g\after f)(g(b)g(b'))=1_A(g(b)g(b'))=g(b)g(b')\]
Therefore \(g\)~is a homomorphism and hence \(f\)~is an isomorphism.
\item[(c)] As in~(b), a poset homomorphism is, in particular, a function, hence an isomorphism is a bijective homomorphism by~(a). However, unlike in~(b), the inverse of a bijective homomorphism need not be a homomorphism. For example, consider a poset consisting of two copies of \(\N=(N,\le)\) with no relations between the copies. Map this poset into~\(\N\) by ``zipping'' the two copies together, sending one to the evens in order, and the other to the odds in order. This mapping is clearly a bijective homomorphism, but its inverse is not since, for example, \(0\le 1\) in the image, but the preimage of~\(0\) is not related to the preimage of~\(1\).\qedhere
\end{enumerate}
\end{proof}
\end{exer}

\begin{exer}[5]
Let \(\cat{C}\)~be a category and \(C\in\cat{C}\). Let \(U:\cat{C}/C\to\cat{C}\) ``forget about the base object~\(C\)'' by mapping each object \(f:A\to C\) to its domain~\(A\) and each arrow \(a:A\to B\) to ``itself.''\footnote{An arrow in~\(\cat{C}/C\) is understood as a triple \((a,f,f')\) where \(a:A\to B\), \(f:A\to C\), and \(f':B\to C\) are arrows in~\(\cat{C}\) with \(f=f'\after a\). So \(U((a,f,f'))=a\).} Then \(U\)~is a functor.

Let \(F:\cat{C}/C\to\arr{\cat{C}}\) map objects to themselves and each arrow \(a:A\to B\) to the pair~\((a,1_C)\), where \(1_C\)~is the identity arrow for~\(C\) in~\(\cat{C}\). Then \(F\)~is a functor, and \(\fdom\after F=U\), where \(\fdom:\arr{\cat{C}}\to\cat{C}\) is the functor mapping each object \(f:A\to B\) to its domain~\(A\) and each arrow \((g_1,g_2)\) to~\(g_1\).
\end{exer}
\begin{proof}
\(U\)~maps objects to objects and arrows to arrows, and preserves domains and codomains of arrows. Since \(\cat{C}/C\) inherits identities and composites from~\(\cat{C}\), \(U\)~also preserves identities and composites. Therefore \(U\)~is a functor.

\(F\)~maps objects to objects and arrows to arrows, and preserves domains and codomains of arrows, since if \(a:A\to B\) maps \(f:A\to C\) to \(f':B\to C\) in~\(\cat{C}/C\), then \(1_C\after f=f=f'\after a\), hence \((a,1_C)\)~maps~\(f\) to~\(f'\) in~\(\arr{\cat{C}}\). Since \(\arr{\cat{C}}\)~also inherits identities and composites from~\(\cat{C}\), \(F\)~also preserves identities and composites. Therefore \(F\)~is a functor. Clearly \(\fdom\after F=U\).
\end{proof}

\begin{exer}[6]
Let \(\cat{C}\)~be a category and \(C\in\cat{C}\). Then \(C/\cat{C}=\dual{\cat{C}}/C\).\footnote{In this exercise, we informally identify an arrow \(f:A\to B\) in~\(\cat{C}\) with the corresponding reversed arrow \(\star{f}:\star{B}\to\star{A}\) in~\(\dual{\cat{C}}\), and with any corresponding arrows in slices or coslices.}
\end{exer}
\begin{proof}
The arrows out of~\(C\) in~\(\cat{C}\) are precisely the arrows into~\(C\) in~\(\dual{\cat{C}}\), and the commuting triangles among the former in~\(\cat{C}\) are precisely the commuting triangles among the latter in~\(\dual{\cat{C}}\). Therefore \(C/\cat{C}\)~and~\(\dual{\cat{C}}/C\) have the same objects and arrows. They also have the same identities and composites since these are inherited from \(\cat{C}\)~and~\(\dual{\cat{C}}\), respectively, where they are the same by definition of~\(\dual{\cat{C}}\).
\end{proof}
\begin{rmk}
Similarly \(C/\dual{\cat{C}}=\cat{C}/C\).
\end{rmk}

\begin{exer}[8]
For a (small) category~\(\cat{C}\), let \(P(\cat{C})\) consist of the objects from~\(\cat{C}\) ordered as follows:
\begin{center}
\(A\le B\) if and only if there exists an arrow \(A\to B\) in~\(\cat{C}\)
\end{center}
Then \(P(\cat{C})\)~is a preorder, and \(P:\Cat\to\Pre\) determines a functor with \(P\after C=1_{\Pre}\), where \(C:\Pre\to\Cat\) is the evident inclusion functor.
\end{exer}
\begin{proof}
Reflexivity and transitivity of the order in~\(P(\cat{C})\) follow from the existence of identities and composites in~\(\cat{C}\). So \(P\)~maps categories to preorders. For a functor \(F:\cat{C}\to\cat{D}\), let \(P(F)\)~be the restriction of~\(F\) to objects. If \(A\le B\) in~\(P(\cat{C})\), there exists an arrow \(f:A\to B\) in~\(\cat{C}\). But then \(F(f):F(A)\to F(B)\) is an arrow in~\(\cat{D}\), so \(F(A)\le F(B)\) in~\(P(\cat{D})\). Therefore \(P(F)\)~is monotone, and hence \(P\)~maps functors to preorder homomorphisms. Since \(P\)~just restricts functors to objects, it preserves domains and codomains, identities, and composites, hence it is a functor. It is obvious that \(P\after C=1_{\Pre}\).
\end{proof}
\begin{rmk}
In general \(C\after P\ne 1_{\Cat}\) because \(P\)~loses information about the arrow structure of categories. Specifically, multiple arrows from one object to another will be represented by a single relation between those objects under~\(P\).
\end{rmk}

\begin{exer}[11]
There exists a functor \(M:\Sets\to\Mon\) mapping each set~\(A\) to the free monoid on~\(A\).
\end{exer}
\begin{proof}
We prove this in two ways.
\begin{enumerate}[itemsep=0pt]
\item[(a)] Let \(M(A)=\star{A}\) and for \(f:A\to B\) define \(M(f):\star{A}\to\star{B}\) by
\[M(f)(a_1\cdots a_k)=f(a_1)\cdots f(a_k)\quad a_1,\ldots,a_k\in A\]
\(M(f)\)~is well defined on~\(\star{A}\) since every element in~\(\star{A}\) can be expressed uniquely as a product of elements of~\(A\), and by construction \(M(f)\)~is a monoid homomorphism extending~\(f\). So \(M\)~maps objects to objects and arrows to arrows. Clearly \(M\)~preserves domains and codomains of arrows and \(M(1_A)=1_{\star{A}}\). If \(g:B\to C\), then
\begin{align*}
M(g\after f)(a_1\cdots a_k)&=(g\after f)(a_1)\cdots(g\after f)(a_k)\\
	&=g(f(a_1))\cdots g(f(a_k))\\
	&=M(g)(f(a_1)\cdots f(a_k))\\
	&=M(g)(M(f)(a_1\cdots a_k))\\
	&=(M(g)\after M(f))(a_1\cdots a_k)
\end{align*}
So \(M(g\after f)=M(g)\after M(f)\) and \(M\)~preserves composites. Therefore \(M\)~is a functor.
\item[(b)] Let \(M(A)\)~be ``the'' free monoid on~\(A\) satisfying the universal mapping property (Propositions 1.9~and~1.10). For \(f:A\to B\to M(B)\), let \(M(f)\)~be the unique monoid homomorphism from~\(M(A)\) to~\(M(B)\) extending~\(f\). Clearly \(M\)~maps objects to objects and arrows to arrows, and preserves domains and codomains of arrows. Now \(1_{M(A)}\)~extends~\(1_A\), hence we must have \(M(1_A)=1_{M(A)}\). Similarly if \(g:B\to C\to M(C)\), then \(M(g)\after M(f)\) extends~\(g\after f\), hence we must have \(M(g\after f)=M(g)\after M(f)\). Therefore \(M\)~is a functor.\qedhere
\end{enumerate}
\end{proof}

% References
\begin{thebibliography}{0}
\bibitem{awodey10} Awodey, Steve. \textit{Category Theory, 2nd~ed.} Oxford, 2010.
\end{thebibliography}
\end{document}